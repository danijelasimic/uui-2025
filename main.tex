\documentclass[a4paper,12pt]{article}
\usepackage[T2A]{fontenc}
\usepackage[utf8]{inputenc}
\usepackage[serbian]{babel}
\usepackage{amsmath, amsthm, amssymb}
\usepackage{xcolor}
\usepackage{graphicx}
\usepackage{wrapfig}
\usepackage{booktabs}
\usepackage{hyperref}
\usepackage{caption}

\newtheorem{teorema}{Теорема}[section]
\newtheorem{definicija}{Дефиниција}[section]
\newtheorem{lema}{Лема}[section]

\title{Конгруенције}
\author{Лука Мијатовић}
\date{\today}

\begin{document}

\maketitle
\tableofcontents
\newpage

\section{Увод у конгруенције}
Теорија конгруенција представља један од темеља модерне теорије бројева. Увео ју је Карл Фридрих Гаус у свом делу \textit{Disquisitiones Arithmeticae}.

\begin{definicija}
Нека је $n \in \mathbb{N}$. Кажемо да су цели бројеви $a$ и $b$ \textbf{конгруентни по модулу} $n$ ако $n$ дели њихову разлику $a - b$. То записујемо као:
$$ a \equiv b \pmod{n} $$
\end{definicija}

\begin{wrapfigure}[12]{l}{0.4\textwidth}
    \centering
    \includegraphics[width=0.35\textwidth]{wilson.jpeg} 
    \caption*{Џон Вилсон (1741–1793)}
\end{wrapfigure}

\section{Вилсонова теорема}
Вилсонова теорема даје неопходан и довољан услов да број буде прост.

\begin{lema}
Ако је $p$ прост број, тада је једини елемент $a \in \{1, 2, \dots, p-1\}$ који је сам себи инверзан по модулу $p$ (тј. $a^2 \equiv 1 \pmod{p}$) заправо $1$ или $p-1$.
\end{lema}

\begin{teorema}[Вилсонова теорема]
Природан број $p > 1$ је прост ако и само ако важи:
\begin{equation}
(p-1)! \equiv -1 \pmod{p}
\end{equation}
\end{teorema}

\leavevmode

\subsection{Примена и примери}
Примена ове теорије је огромна, посебно у \textbf{\textcolor{red}{криптографији}} и рачунарству.

\begin{itemize}
    \item Провера простности бројева.
    \item Генерисање \textit{RSA} кључева.
    \item Теорија група.
\end{itemize}

\section{Табеларни приказ остатака}
У следећој табели приказани су остаци при дељењу са малим простим бројевима за вредност $(p-1)!$.

\begin{table}[h]
\centering
\begin{tabular}{|c|c|r|}
\hline
\textbf{Број $p$} & \textbf{Формула $(p-1)!$} & \textbf{Остатак по модулу $p$} \\ \hline
2 & $1! = 1$ & $1 \equiv -1 \pmod{2}$ \\ \hline
3 & $2! = 2$ & $2 \equiv -1 \pmod{3}$ \\ \hline
5 & $4! = 24$ & $24 \equiv -1 \pmod{5}$ \\ \hline
\end{tabular}
\caption*{Провера Вилсонове теореме за мале бројеве}
\end{table}

\section{Закључак}
\textsf{Конгруенције нам омогућавају да на елегантан начин решавамо сложене проблеме дељивости. Иако је Вилсонова теорема од великог теоријског значаја, у пракси се за велике бројеве чешће користе други алгоритми због факторијела који брзо расте.}

\begin{enumerate}
    \item Први корак: Разумевање дељивости.
    \item Други корак: Примена на просте бројеве.
\end{enumerate}

\end{document}
