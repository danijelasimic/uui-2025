\documentclass[12pt,a4paper]{article}

\usepackage[T2A]{fontenc}
\usepackage[utf8]{inputenc}
\usepackage[serbian]{babel}

\usepackage{geometry}
\geometry{margin=2.5cm}

\usepackage{amsmath,amssymb,amsthm}
\usepackage{graphicx}
\usepackage{float}

\addto\captionsserbian{
  \renewcommand{\figurename}{Слика}
  \renewcommand{\tablename}{Табела}
}

\theoremstyle{definition}
\newtheorem{definicija}{Дефиниција}
\newtheorem{lema}{Лема}
\theoremstyle{plain}
\newtheorem{teorema}{Теорема}

\title{\textbf{Едвард Сноуден и приватност у информационим системима}}
\author{Никола Радосављевић}
\date{јануар 2026.}

\begin{document}

\maketitle
\newpage
\tableofcontents
\newpage

\section{Увод}
Развој информационих система омогућио је обраду и анализу великих количина података.
Истовремено, све израженији надзор корисника покренуо је питања приватности и етичке употребе технологије.
Случај Едварда Сноудена из 2013. године представља један од најпознатијих примера у којем су ови проблеми
доспели у жижу јавности.

\section{Подаци и надзор}
Савремени системи прикупљају различите врсте података како би обезбедили функционалност,
поузданост и безбедност. Комбинација прикупљених информација омогућава детаљну анализу
активности корисника, што уједно представља и потенцијални ризик по приватност.

У том контексту, појам приватности може се формално посматрати на следећи начин.

\begin{definicija}
Приватност у информационим системима означава степен контроле корисника над прикупљањем,
обрадом и употребом његових података.
\end{definicija}

\subsection{Учесталост приступа}
Један од индикатора активности у систему јесте учесталост приступа.
Ако је у периоду од $T$ сати забележено $N$ приступа,
просечна учесталост може се описати изразом
\[
\lambda = \frac{N}{T}.
\]
Овај модел се користи приликом анализе логова и откривања необичних образаца понашања.

\section{Категорије података}
Приликом надзора разликују се различите категорије информација,
које заједно могу пружити веома прецизну слику о активностима корисника.

\begin{table}[H]
\centering
\caption{Типичне категорије података у системима надзора}
\begin{tabular}{|l|l|l|}
\hline
Категорија & Пример & Улога \\
\hline
Садржај & текст поруке & директна информација \\
Метаподаци & време, контакти & обрасци понашања \\
Технички подаци & IP адреса & контрола и безбедност \\
\hline
\end{tabular}
\end{table}

\begin{figure}[H]
\centering
\includegraphics[width=0.35\textwidth]{slike/slika.png}
\caption{Симбол заштите приватности}
\end{figure}

\section{Мере заштите}
Како би се умањили ризици, у пракси се примењују различите мере,
укључујући шифровање комуникације, ограничење приступа подацима и евиденцију активности.
Ефекат оваквих мера може се сажети следећим запажањем.

\begin{lema}
Систем у коме су приступи подацима ограничени и евидентирани
има мањи ризик од неовлашћене употребе информација.
\end{lema}

На основу овог запажања може се формулисати и општије тврђење.

\begin{teorema}
Информациони системи са јасно дефинисаним правилима приступа
обезбеђују виши ниво заштите приватности корисника.
\end{teorema}

\section{Закључак}
Случај Едварда Сноудена указао је на значај транспарентности и одговорне обраде података.
Са становишта информатике, приватност представља неодвојив део пројектовања
и одржавања савремених информационих система.

\end{document}
