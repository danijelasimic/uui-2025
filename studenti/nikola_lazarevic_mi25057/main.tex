\documentclass[a4paper,12pt]{article}
\usepackage[utf8]{inputenc}
\usepackage[T2A]{fontenc}
\usepackage[serbian]{babel}
\usepackage{amsmath, amsthm}
\usepackage{graphicx}
\usepackage{xcolor}
\usepackage{float}
\usepackage{wrapfig}

\newtheorem{teorema}{Теорема}
\newtheorem{definicija}{Дефиниција}
\newtheorem{lema}{Лема}

\title{\textbf{Криптографија – основе за почетнике}}
\author{Никола Лазаревић \\ 57/2025}
\date{\today}

\begin{document}

\maketitle
\tableofcontents
\newpage


\section{Увод}
Криптографија представља \textbf{науку о заштити информација} од неовлашћеног приступа. 
У савременом друштву велики број података се размењује путем интернета, па је потребно обезбедити њихову поверљивост и сигурност.



\section{Основни појмови}

\subsection{Дефиниција криптографије}

\begin{definicija}
Криптографија је дисциплина која проучава методе трансформације података у облик који је нечитљив свима осим овлашћеним учесницима комуникације.
\end{definicija}

\subsection{Модел комуникације}

\begin{wrapfigure}{l}{0.4\textwidth}
\centering
\includegraphics[width=0.38\textwidth]{Slike/kriptografija.jpg}
\end{wrapfigure}

Најједноставнији модел подразумева пошиљаоца, поруку и примаоца. 
Додавањем процеса шифровања и дешифровања обезбеђује се сигурност комуникације. 
Овакви системи се данас користе у банкарству, електронској пошти и заштити лозинки.

\hspace{0.1cm}
\section{Типови криптографије}

\subsection{Симетрична}
Користи се један тајни кључ за оба процеса. 
Ова метода је \textbf{веома брза} и погодна за велике количине података.

\subsection{Асиметрична}
Користе се јавни и приватни кључ. Иако је спорија, омогућава већу флексибилност у комуникацији.

\subsection{Карактеристике}

\begin{itemize}
\item заштита поверљивости
\item очување интегритета
\item аутентичност корисника
\end{itemize}

\subsection{Особине}

\begin{enumerate}
\item генерисање кључа
\item шифровање поруке
\item дешифровање поруке
\end{enumerate}

% ------------------------------------------------

\section{Математичка основа}

Криптографија се у великој мери ослања на математику. 
На пример, једноставан приказ функције шифровања може бити:

$${\color{red}C = E(K, P)}$$

где је ${\color{red}P}$ оригинална порука, ${\color{red}K}$ кључ, а ${\color{red}C}$ шифровани текст.

\begin{lema}
Уколико је кључ тајан, нападач без његовог познавања не може у разумном времену доћи до оригиналне поруке.
\end{lema}

\begin{teorema}
Безбедност криптографског система директно зависи од јачине кључа и алгоритма који се користи.
\end{teorema}

% ------------------------------------------------

\section{Примена алгоритама}

\begin{table}[H]
\centering
\begin{tabular}{|c|c|c|}
\hline
Алгоритам & Тип & Примена \\
\hline
AES & симетрични & заштита датотека \\
RSA & асиметрични & дигитални потпис \\
SHA & хеш & лозинке \\
\hline
\end{tabular}
\caption{Примери криптографских алгоритама}
\end{table}

% ------------------------------------------------

\section{Закључак}
Криптографија има кључну улогу у савременом информационом друштву. 
Комбинацијом различитих метода омогућава се висок ниво сигурности података и поверење у дигиталну комуникацију.

\end{document}
