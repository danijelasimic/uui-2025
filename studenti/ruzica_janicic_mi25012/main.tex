\documentclass[a4paper,12pt]{article}
\usepackage[T2A]{fontenc}
\usepackage[utf8]{inputenc}
\usepackage[serbian]{babel}
\usepackage{amsmath,amsthm,amssymb}
\usepackage{graphicx}
\usepackage{wrapfig}
\usepackage{xcolor}
\usepackage{geometry}
\geometry{margin=2.5cm}

\newtheorem{theorem}{Теорема}[section]
\newtheorem{lemma}[theorem]{Лема}
\theoremstyle{definition}
\newtheorem{definition}{Дефиниција}[section]

\title{\textbf{Паскалина: \\ Први механички калкулатор}}
\author{Ружица Јаничић \\ Математички факултет}
\date{Децембар 2025.}

\begin{document}

\maketitle
\renewcommand{\contentsname}{Садржај} 
\tableofcontents
\newpage

\section{Увод}

\textbf{Паскалова машина} (такође позната као \textit{Паскалина}) представља један од најзначајнијих проналазака у историји рачунарства. Направио ју је француски математичар, физичар и филозоф \textcolor{brown}{Блез Паскал} 1642. године, када је имао само 19 година.

\begin{wrapfigure}{l}{0.35\textwidth}
\centering
\includegraphics[width=0.3\textwidth]{paskalina.jpg}
\caption{Паскалина}
\label{fig:paskaline}
\end{wrapfigure}

Главна мотивација за стварање ове машине била је Паскалова жеља да помогнем свом оцу, који је радио као порески инспектор и морао је да обавља мучне нумеричке прорачуне. Машина је сабирала и одузимала бројеве помоћу система зупчаника.

Овај изум представља \textit{револуционаран корак} у аутоматизацији рачунања и поставио је темеље за касније развоје у области механичких и електронских рачунара.

\section{Принцип рада}

\subsection{Механички систем}

Паскалова машина користи систем међусобно повезаних зупчаника за извршавање аритметичких операција. Основни принцип рада заснива се на:

\begin{itemize}
\item Носиоцима — када један зупчаник пређе пун круг, активира се носилац
\item Декадном систему — свака цифра представљена је зупчаником са 10 зубаца
\item Механичком преносу енергије кроз систем
\end{itemize}

\subsection{Математички модел}

Операција сабирања два броја $a$ и $b$ може се математички представити као:

$$S = a + b = \sum_{i=0}^{n} (a_i + b_i) \cdot 10^i + c_i$$

где је $c_i$ носилац  са позиције $i$.

\begin{definition}
\textbf{Носилац} је механизам који се активира када збир цифара на одређеној позицији пређе вредност 9, преносећи јединицу на следећу вишу позицију.
\end{definition}

\section{Техничке карактеристике}

\subsection{Компоненте машине}

У табели \ref{tab:komponente} приказане су главне компоненте Паскалове машине:

\begin{table}[h]
\centering
\begin{tabular}{|l|c|p{7cm}|}
\hline
\textbf{Компонента} & \textbf{Број} & \textbf{Опис} \\
\hline
Зупчаници & 8 & Један за сваку декадну позицију \\
\hline
Носиоци & 7 & Између суседних зупчаника \\
\hline
Бројчаници & 8 & За приказ резултата \\
\hline
Точкићи & 8 & За унос бројева \\
\hline
\end{tabular}
\caption{Главне компоненте Паскалове машине}
\label{tab:komponente}
\end{table}

\subsection{Ограничења}

Паскалова машина имала је неколико значајних ограничења:

\begin{enumerate}
\item Могла је да ради само са целим бројевима
\item Одузимање је било компликовано (коришћен је метод комплемента)
\item Множење и дељење нису били директно подржани
\item Била је скупа за производњу
\end{enumerate}


\section{Закључак}

Паскалина представља један од најважнијих техничких изума 17. века. Иако једноставна у поређењу са модерним рачунарима, она је допринела аутоматизацији рачунања и послужила као инспирација каснијим проналасцима.

\end{document}