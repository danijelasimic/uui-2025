\documentclass[a4paper,12pt]{article}

\usepackage[T2A]{fontenc}
\usepackage[utf8]{inputenc}
\usepackage[serbianc]{babel}
\usepackage{amsmath}
\usepackage{amsthm}
\usepackage{graphicx}
\usepackage{xcolor}
\usepackage{float}
\usepackage{geometry}
\geometry{margin=2.5cm}

\title{\textbf{Како функционише Интернет}}
\author{Димитрије Новаковић \\ Број индекса: 30/2025}
\date{\today}

\newtheorem{definicija}{Дефиниција}
\newtheorem{teorema}{Теорема}
\newtheorem{lema}{Лема}

\begin{document}

\maketitle
\tableofcontents
\newpage

\section{Увод}

\textbf{\textcolor{blue}{Интернет}} представља глобалну комуникациону инфраструктуру
засновану на TCP/IP протоколима.

\section{Основни концепти}

\begin{definicija}
IP адреса је јединствени идентификатор уређаја у мрежи.
\end{definicija}

\begin{lema}
Сваки пакет садржи изворну и одредишну IP адресу.
\end{lema}

\begin{teorema}
TCP протокол гарантује поузданост преноса коришћењем механизма потврде.
\end{teorema}

\subsection{Математички модел преноса}

Количина података:

\[
Q = B \cdot t
\]

где је:
\begin{itemize}
\item $B$ — брзина преноса,
\item $t$ — време преноса.
\end{itemize}

\section{Табела протокола}

\begin{table}[H]
\centering
\begin{tabular}{|c|c|}
\hline
Протокол & Намена \\
\hline
HTTP & Пренос веб садржаја \\
TCP & Поуздан транспорт \\
UDP & Брзи транспорт без гаранције \\
DNS & Превођење домена \\
\hline
\end{tabular}
\caption{Кључни интернет протоколи}
\end{table}

\section{Илустрација мреже}

\begin{figure}[H]
\centering
\includegraphics[width=0.4\textwidth]{slike/mreza.png}
\caption{Структура рачунарске мреже клијент-сервер}
\end{figure}

\section{Закључак}

\textit{Интернет} представља један од најзначајнијих технолошких система
савременог друштва и омогућава глобалну комуникацију.

\end{document}