\documentclass[a4paper]{article}
\usepackage{graphicx}
\usepackage[utf8]{inputenc}
\usepackage[english,serbian]{babel}
\usepackage[OT2]{fontenc}
\usepackage{amsmath}
\usepackage{hyperref}
\usepackage{amsthm}
\usepackage{wrapfig}
\usepackage{xcolor}
\sloppy

\addto\captionsserbian{\renewcommand{\contentsname}{Садржај}}
\newtheorem{teorema}{Теорема}[section]
\newtheorem{lema}{Лема}[section]
\theoremstyle{definition}
\newtheorem{definicija}{Дефиниција}[section]

\title{\textbf{Најнижи заједнички предак два чвора у стаблу}}
\author{Софија Чебашек}
\date{Фебруар 2026.}

\begin{document}

\maketitle

\tableofcontents
\clearpage

\maketitle

\section{Увод}
Дато је стабло са $n$ чворова и његов корен. За два дата чвора, $u$ и $v$, потребно је одредити најнижег заједничког претка (енг. \emph{\textlatin{Lowest Common Ancestor - LCA}}). Овај проблем је могуће решити на више начина, неки од њих су:
\begin{itemize}
    \item Кретањем од два дата чвора навише у временској сложености $O(n)$;
    \item Користећи технику померања за степен двојке тј. технику бинарног скока (енг. \emph{\textlatin{binary lifting}}) у временској сложености $O(\log n)$.
\end{itemize}

\begin{wrapfigure}{l}{0.4\textwidth}
    \includegraphics[scale=0.45]{graf.png}
    \caption{Пример стабла}
    \label{graf}
    \vspace{-1cm}
\end{wrapfigure}

\section{Алгоритам}
За дато стабло са $n$ чворова и кореном, за два његова дата чвора неопходно је одредити чвор који је најдаљи од корена, а предак је оба дата чвора.

\subsection{Основни појмови}
\begin{definicija}
    Чвор $p$ је \textbf{предак} чвора $u$ уколико се налази на путу од чвора $u$ до корена стабла. 
\end{definicija}
\begin{definicija}
    Чвор $l$ је \textbf{заједнички предак} чворова $u$ и $v$, уколико је предак оба чвора. Уколико је $l$ на највећој удаљености од корена стабла од свих заједничких предака, тада је он \textbf{најнижи заједнички предак}.
\end{definicija}
На слици \textcolor{purple}{\ref{graf}} корен стабла је чвор 1. Тада је најнижи заједнички предак чворова 8 и 9 чвор 7, чворова 5 и 6 чвор 1, чворова 3 и 4 чвор 1 итд.
\subsection{Идеја алгоритма}
\begin{definicija}
    Нека је $m=\lceil \log n \rceil$. Матрица $predak[n][m]$ у пољу $predak[i][j]$ садржи претка чвора $i$ који се налази на растојању $2^j$, где растојање између два чвора представља број грана на путу између њих. У случају да такав предак не постоји, када је чвор $i$ на растојању од корена мањем од $2^j$, у пољу $predak[i][j]$ биће сачувана вредност корена стабла.
\end{definicija}
\begin{lema}\label{lema1}
Важи $predak[i][j]=predak[predak[i][j-1]][j-1]$.    
\end{lema}
\begin{definicija}
    У низу $ulaz[n]$, елемент $ulaz[i]$ представља време прве посете, односно време уласка у подстабло чвора $i$ приликом обиласка стабла, док у низу $izlaz[n]$, елемент $izlaz[i]$ представља време изласка из чвора $i$ и његовог подстабла.
\end{definicija}
Користећи лему \textcolor{purple}{\ref{lema1}} и чињеницу да је $predak[i][0]$ родитељ чвора $i$, једним обиласком стабла (алгоритмом претраге у дубину) могуће је израчунати све елементе матрице $predak[n][m]$. Приликом обиласка стабла рачунаћемо и низове $ulaz[n]$ и $izlaz[n]$.
\begin{teorema}\label{teorema1}
    Чвор $u$ је предак чвора $v$ уколико важи $ulaz[u] \leq ulaz[v]$ и $izlaz[u] \geq izlaz[v]$.
\end{teorema}
Након проласка кроз стабло, прво се провери да ли је неки од два дата чвора $u$ и $v$ предак другог. Уколико јесте, тај чвор представља њиховог најнижег заједничког претка. Уколико није, тражи се највиши предак чвора $u$, који није предак чвора $v$. Од чвора $u$ се врши померање за највећи број облика $2^i$, тако да $predak[u][i]$ није предак чвора $v$. Вредност $i$ је на почетку $m-1$ и смањује се до $0$, а уколико $predak[u][i]$ није предак чвора $v$, нови чвор $u$ је чвор $predak[u][i]$ и алгоритам се наставља даље. Након $m$ корака, чвор $u$ ће имати вредност највишег чвора који није заједнички  предак, а његов родитељ ће бити тражени чвор тј. чвор $predak[u][0]$ је тражени најнижи заједнички предак.

\begin{table}
    \begin{tabular}{|c|c|}
        \hline
        \textbf{Корак} &  \textbf{Временска сложеност}\\
        \hline
        Иницијализација почетних низова & $O(n \log n)$\\
        \hline
        Провера да ли је један чвор предак другог & $O(1)$\\
        \hline
        Највиши чвор који није заједнички предак & $O(\log n)$\\
        \hline
        \hline
        \textbf{Сложеност алгоритма за један упит} & $O(\log n)$\\
        \hline
        \textbf{Укупна сложеност алгоритма за $q$ упита} & $O((n+q) \log n)$\\
        \hline
    \end{tabular}
    \caption{Анализа временске сложености алгоритма}
\end{table}

\section{Закључак}
Укупна сложеност описаног алгоритма за стабло са $n$ чворова и $q$ упита који су задати као пар чворова је $O((n+q) \log n)$. Једна од честих употреба алгоритма за одређивање најнижег заједничког претка је приликом одређивања растојања између два чвора стабла.

\end{document}
