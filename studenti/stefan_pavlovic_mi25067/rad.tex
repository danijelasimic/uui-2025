\documentclass{article}
\usepackage[T2A]{fontenc}
\usepackage[serbianc]{babel}
\usepackage{graphicx,wrapfig}
\usepackage{subfig}
\usepackage{url}
\usepackage{xcolor}
\usepackage{amsthm}
\usepackage{amssymb}
\usepackage{float}

\theoremstyle{plain}
\newtheorem{theorem}{Теорема}[section]
\newtheorem{lemma}[theorem]{Лема}

\theoremstyle{definition}
\newtheorem{definition}[theorem]{Дефиниција}

\usepackage[unicode]{hyperref}
\hypersetup{colorlinks,citecolor=green,filecolor=green,linkcolor=blue,urlcolor=blue}

\title{Крускалов Алгоритам}
\author{Стефан Павловић}
\date{2026}

\begin{document}

\maketitle

\tableofcontents
\newpage

\section{Увод}
Крускалов алгоритам проналази минимално разапињуће стабло за повезан тежински граф. Уколико је граф није повезан алгоритам проналази минималну разапињућу шуму. Временска комплексност му је $O(E \log V)$. Алгоритам је открио Џозеф Крускал. 
\par
Примене минималног разапињућег стабла:
\begin{itemize}
    \item Постављање каблова
    \item Рутирање путања у мрежама
    \item Мрежа испорука
    \item Пројектовање путева
    \item Кластеризација података
    \item Многе друге
\end{itemize}
\section{Крускалов алгоритам}

\begin{definition}[Минимално разапињуће стабло]
Нека је $G = (V, E)$ повезан тежински граф. Разапињуће стабло $T = (V, E_T)$ графа $G$ се назива \textbf{минимално разапињуће стабло} ако за свако друго разапињуће стабло $T' = (V, E_{T'})$ важи:
$$w(T) = \sum_{e \in E_T} w(e) \leq \sum_{e \in E_{T'}} w(e) = w(T')$$
\end{definition}

\begin{lemma}[Лема о сигурној грани]
Нека је $A$ скуп грана које припадају неком минималном разапињућем стаблу графа $G$. Ако грана $e$ минималне тежине повезује две различите компоненте графа $(V, A)$, онда $A \cup \{e\}$ такође припада неком минималном разапињућем стаблу.
\end{lemma}

\begin{theorem}[Коректност Крускаловог алгоритма]
Крускалов алгоритам производи минимално разапињуће стабло за сваки повезан тежински граф.
\end{theorem}

Како алгоритам функционише:

\begin{enumerate}
    \item Сортирају се све ивице графа у неопадајућем редоследу
    \item Итерирамо кроз ивице графа, уколико додавање ивице у стабло не прави циклус, додамо ивицу
    \item На крају добијемо минимално разапињућу шуму уколико граф није повезан, односно минимално разапињуће стабло уколико је граф повезан
\end{enumerate}

\subsection{Други алгоритми минималног разапињућег стабла}
Неки други алгоритми минималног разапињућег \textcolor{blue}{стабла} (\textit{шуме}) су:

\begin{table}[H]
\begin{center}
\begin{tabular}{|c|c|c|}
\hline
Алгоритам & Временска комплексност & Година открића \\
\hline
Борувкин алгоритам & $O(E \log V)$ & 1926 \\
\hline
Примов алгоритам & $O(E \log V)$ & 1957 \\
\hline
Алгоритам обрнутог брисања & \textcolor{red}{$O(E \log V (\log \log V)^3)$} & 1956 \\
\hline
\end{tabular}
\end{center}
\end{table}

Једна занимљивост је да је алгоритам обрнутог брисања такође открио \textbf{Џозеф Крускал}.

\subsection{Џозеф Бернард Крускал млађи}

\begin{wrapfigure}{r}{3.5cm}
\begin{center}
\includegraphics[width=3.5cm]{slike/kruskal.jpeg}
\end{center}
\caption{Џозеф Бернард Крускал млађи}
\label{fig:for_disk}
\end{wrapfigure}

Џозеф Бернард Крускал млађи (1928-2010) је био Амерички математичар. Основне и мастер студије је завршио на универзитету у Чикагу док је докторирао на универзитету Принстон. Радио је у Беловој лабораторији од 1959. до 1993. године.

\section{Закључак}
Крускалов алгоритам је један од најзначајнијих алгоритама за проналажење минималног разапињућег стабла и има широк спектар примена.

\end{document}
