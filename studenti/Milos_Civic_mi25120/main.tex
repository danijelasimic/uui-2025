\documentclass[12pt,a4paper]{article}
\usepackage{fontspec}
\usepackage{polyglossia}
\setmainlanguage{serbian}
\setmainfont{Times New Roman}

\usepackage{amsmath, amsthm}
\usepackage{graphicx}
\usepackage{xcolor}
\usepackage{hyperref}

% Definicije okruženja
\newtheorem{teorema}{Теорема}
\newtheorem{definicija}{Дефиниција}
\newtheorem{lema}{Лема}

\title{Семинарски рад из Увода у информатику}
\author{Милош, број индекса MI25120}
\date{\today}

\begin{document}

\maketitle
\tableofcontents
\newpage

\section{Увод}
Добро дошли у мој академски текст. Овај рад је написан у \LaTeX-у и служи као демонстрација основних могућности: форматирање текста, математичке формуле, табеле, листе и уметање слика.  

\textbf{Циљ} је да се прикаже како се користе основни елементи.  
\textit{Напомена:} текст је на ћирилици.  
\textcolor{blue}{Ово је пример текста у плавој боји.}

\section{Математика}
\subsection{Формуле}
Једна од најпознатијих формула у физици је:
\[
E = mc^2
\]
где је $E$ енергија, $m$ маса, а $c$ брзина светлости.

\subsection{Теорема, дефиниција и лема}
\begin{definicija}
Скуп је основни појам математике који представља колекцију објеката.
\end{definicija}

\begin{teorema}
Ако је $A \subseteq B$ и $B \subseteq C$, онда је $A \subseteq C$.
\end{teorema}

\begin{lema}
За сваки природан број $n$ важи $n < n+1$.
\end{lema}

\section{Табела и листе}
\subsection{Табела}
Пример једноставне табеле:

\begin{center}
\begin{tabular}{|c|c|}
\hline
Семестар & Предмети \\ \hline
1. & Увод у програмирање, Увод у информатику, Линеарна алгебра \\ \hline
2. & Анализа 1, Увод у алгоритме, Архитектура рачунара \\ \hline
\end{tabular}
\end{center}

\subsection{Листе}
Нумерисана листа:
\begin{enumerate}
    \item Први елемент
    \item Други елемент
    \item Трећи елемент
\end{enumerate}

Ненумерисана листа:
\begin{itemize}
    \item Један
    \item Два
    \item Три
\end{itemize}

\section{Слика}
На слици испод је симболичан пример (може бити иста као у HTML делу):

\begin{figure}[h]
\centering
\includegraphics[width=0.45\textwidth]{slike/profil.jpg}
\caption{Профилна фотографија — пример уметања слике}
\end{figure}

\section{Закључак}
У овом раду приказали смо основне могућности \LaTeX-а: форматирање текста, математичке формуле, табеле, листе, слике и дефинисање теорема.  

\textbf{Закључак:} \LaTeX{} је моћан алат за академско писање и омогућава јасну структуру и професионалан изглед текста.  
\textit{Додатно:} овај рад показује да се уз мало труда може направити комплетан академски документ који испуњава све задате критеријуме.

\end{document}