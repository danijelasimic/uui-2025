\documentclass[12pt, a4paper]{article}
\usepackage[T2A]{fontenc}
\usepackage[utf8]{inputenc}
\usepackage[serbian]{babel} 
\usepackage{geometry}  
\usepackage{graphicx}
\usepackage{titlesec}
\usepackage[table]{xcolor}
\usepackage{colortbl}
\usepackage{booktabs}             
\usepackage{amsmath, amssymb, amsthm}  
\usepackage[colorlinks=true, linkcolor=blue, urlcolor=blue]{hyperref}

\graphicspath{{slike/}}

\definecolor{pastelblue}{RGB}{230,242,255}

\definecolor{darkorrange}{RGB}{255,140,0}
\definecolor{darkblue}{RGB}{10,10,255}

\definecolor{pastelblueA}{RGB}{230,242,255} % svetlija plava
\definecolor{pastelblueB}{RGB}{210,230,255} % tamnija plava

\newtheorem{definicija}{Дефиниција}[section]


\geometry{
 a4paper,
 left=25mm,
 right=25mm,
 top=25mm,
 bottom=25mm
}

\title{Модели за прогнозу времена}
\author{Taња Тошаковић}
\date{\today}
\begin{document}


\renewcommand{\contentsname}{Садржај}
\renewcommand{\tablename}{Taбела}
\renewcommand{\figurename}{Слика}


\maketitle
\thispagestyle{empty}
\tableofcontents
\newpage

\section{Нумерички модели}
Нумерички модели за прогнозу времена засновани су на физичким једначинама које описују стање атмосфере. Због присуства нелинеарних чланова, ове једначине је веома тешко, а често и немогуће, решити аналитичким путем. Због тога се примењују нумеричке методе, при чему се изводи замењују коначним разликама, а континуалне величине дискретним вредностима. У том процесу се губи један део информација. 

Атмосфера је у нумеричким моделима представљена мрежом тачака, при чему су вредности метеоролошких променљивих познате искључиво у тим тачкама. Вредности између тачака добијају се применом нумеричких и статистичких метода, као што су интерполација и линеаризација.
Прогнозе се заснивају на \textit{почетним условима} добијеним мерењима са сателита, радара и метеоролошких станица. 

\subsection{Основне једначине атмосфере}
\subsubsection{Једначине кретања}
\begin{align}
u_t + u u_x + v u_y + w u_z - fv &= -\frac{1}{\rho} \frac{\partial p}{\partial x} + F_x \\
v_t + u v_x + v v_y + w v_z + fu &= -\frac{1}{\rho} \frac{\partial p}{\partial y} + F_y \\
w_t + u w_x + v w_y + w w_z + f'u&= -\frac{1}{\rho} \frac{\partial p}{\partial z} - g + F_z
\end{align}

\noindent где је:  
\begin{description}
    \item $u, v, w$ --- компоненте брзине ваздуха у правцима $x, y, z$  
    \item $u_t, v_t, w_t$ --- парцијални изводи по времену (\(\partial u/\partial t\))  
    \item $u_x, u_y, u_z, \dots$ --- парцијални изводи по просторним координатама  
    \item $f$ --- компонента Кориолисове силе у вертикалном правцу, Корилисов фактор (\(2 \Omega \sin \varphi\), где је \(\varphi\) географска ширина)  
    \item $f'$ --- компонента Кориолисове силе у меридионалном правцу (\(2 \Omega \cos \varphi\)) 
    \item $p$ --- атмосферски притисак  
    \item $\rho$ --- густина ваздуха  
    \item $g$ --- гравитациона константа  
    \item $F_x, F_y, F_z$ --- компоненте силе трења/вискозности
\end{description}


\subsubsection{Једначина континуитета(закон очувања масе)}
\begin{align}
\frac{1}{\rho}\frac{\partial \rho}{\partial t} + \frac{\partial u}{\partial x} + \frac{\partial v}{\partial y} + \frac{\partial w}{\partial z} = 0
\end{align}

\noindent где је:  
\begin{description}
    \item $\rho$ --- густина ваздуха  
    \item $u, v, w$ --- компоненте брзине ваздуха  
\end{description}

\subsubsection{Први принцип термодинамике(закон о одржању енергије)}
\begin{align}
\frac{dT}{dt} = \frac{Q}{c_p} - \frac{R T}{p} \frac{dp}{dt}
\end{align}

\noindent где је:  
\begin{description}
    \item $T$ --- температура ваздуха  
    \item $Q$ --- извор или губитак топлоте (нпр. соларна радијација)  
    \item $c_p$ --- специфична топлота при сталном притиску  
    \item $R$ --- гасна константа за суви ваздух  
    \item $p$ --- атмосферски притисак
\end{description}

\subsubsection{Једначина стања идеалног гаса}
\begin{align}
p = \rho R T
\end{align}

\noindent gde je:
\begin{description}
    \item $p$ --- атмосферски притисак  
    \item $\rho$ --- густина ваздуха  
    \item $R$ --- гасна константа за суви ваздух  
    \item $T$ --- температура ваздуха
\end{description}

\subsection{Извори грешака}
\begin{itemize}
  \item Недовољно прецизна мерења( мерни инструменти увек имају грешку)
  \item Недовољан број и покривеност мерења(немамо податке са сваке тачке Земље и у сваком временском тренутку)
  \item Недовољно разложени процеси( удаљеност тачака мреже је превелика да би се процес представио, нпр. турбуленција, или процес није довољно схваћен да би се прецизно представио)
  \item Хаотичност атмосфере( мале грешке се умножавају временом)
  \item Дискретизација простора и времена( вредности имамо само у тачкама мреже, па се између тачака добијају приближне вредности, што уноси грешку)
\end{itemize}

\subsection{Глобални и регионални модели}

Нумерички модели могу бити глобални(покривају целу Землљу) или регионалне(поркивају део Земље). Глобални модели се користе за прогнозу почетних и бочних граничних услова за регионалне моделе. Регионалне модели због мање територије који покрива мрежа могу имати тачке на мањој удаљености, односно имају вишу резолуцију од глобалних. То омогућава боље разлагање физичких процеса, нпр. конвективних процеса, и прецизније локалне прогнозе. 

\begin{table}[h!]
\centering
\rowcolors{2}{pastelblueA}{pastelblueB}
\begin{tabular}{l l}
\toprule
\textbf{Глобални модели} & \textbf{Регионални модели} \\
\midrule
Цела Земља & Ограничено подручје \\
Нижа просторна резолуција & Виша просторна резолуција \\
Не захтева бочне граничне услове & Захтева бочне услове из глобалног модела \\
Велика количина података & Детаљни локални подаци \\
\bottomrule
\end{tabular}
\caption{Поређење глобалних и регионалних нумеричких модела}
\end{table}

\section{АИ/МЛ модели}

Код класичних модела машинског учења, човек унапред дефинише облик функције којом се повезују улазне и излазне променљиве. На пример, избором линеарне регресије, као алгоритма који тражи везу између улазних и излазних променљивих, претпоставља се да међу променљивама постоји линеарна веза. Модел тада из података учи само вредности параметара (коефицијената) те функције. Другим речима, човек наговештава какву везу треба тражити, а модел проналази колико је она јака.

Насупрот томе, модели дубоког учења не ослањају се на унапред зададен облик функције. Они истовремено уче и облик и параметре функције која најбоље описује релације у подацима. Тимe дубоко учење превазилази границе класичног ML приступа и омогућава моделима да самостално откривају сложене, нелинеарне образце и односе, што је посебно значајно у хидролошким системима где су процеси више пута зависни и просторно-временски повезани.

AI модели траже статистичке везе између променљивих, а у зависности од типа односа који желимо да откријемо, бира се архитектура модела.

Посебно обећавајући приступи укључују интеграцију традиционалних нумеричких модела и АИ метода у хибридне системе, који комбинују физичко разумевање процеса и снагу предиктивне анализе података.

\begin{definicija}
\color{darkorrange}{Вештачка интелигенцјија (AI – Artificial Intelligence) представља шири оквир свих метода којима рачунар може да покаже „паметно“ понашање — да учи, препознаје обрасце и доноси одлуке.}
\end{definicija}

\begin{definicija}
\color{darkblue}{Машинско учење (ML – Machine Learning) чини подскуп AI, до ке дубоко учење (DL – Deep Learning) подскуп ML.}
\end{definicija}

\begin{figure}[h!]
    \centering
    \includegraphics[width=0.8\textwidth]{slike/hibridni.png}
    \caption{Илустрација хибридног приступа}
\end{figure}




\section*{Закључак}
\textit{Традиционални нумерички модели, иако веома корисни, суачавају се са ограничењима везаним за сложеност процеса и доступност улазних података. Развој вештачке интелигенције отворио је нове могућности у прогнози времена, захваљујући способности да анализира велике количине података и препознаје нелинеарне образце.
AI приступи, посебно дубоко учење, омогућавају прецизније прогнозе у реалном времену, док \textbf{хибридни модели} комбинују предности физичког разумевања процеса и адаптивности алгоритама.}

\newpage

\section*{Литература}
\begin{enumerate}
  \item Каратина Вељовић Корачанин и Немања Ковачевић, \textit{Општа меторологија}, Београд: АGM knjiga , 2024.
  \item Лазар Лазић \textit{Прогноза времена}, Београд: Републички хирдометеоролошки завод, 2010.
  \item https://medium.com
  \item https://www.geeksforgeeks.org
  \item https://learn.microsoft.com/en-us
\end{enumerate}

\addcontentsline{toc}{section}{Литература}

\end{document}