\documentclass[a4paper,12pt]{article}

\usepackage[T2A]{fontenc}
\usepackage[utf8]{inputenc}
\usepackage[serbian]{babel}
\usepackage{tocloft}
\usepackage{xcolor}
\usepackage{amsmath, amsthm, amssymb}
\usepackage{graphicx}
\usepackage{wrapfig}
\usepackage{booktabs}
\usepackage[hidelinks]{hyperref}

\addto\captionsserbian{\renewcommand{\contentsname}{Садржај}}

\definecolor{myblack}{RGB}{0,0,0}
\renewcommand{\cfttoctitlefont}{\Large\bfseries}
\renewcommand{\cftaftertoctitle}{\hfill}
\patchcmd{\tableofcontents}{\contentsname}{\hfill\contentsname\hfill}{}{}

\title{\textbf{Историја рачунарства}}
\author{Лана Петровић}
\date{\today}

\theoremstyle{definition}
\newtheorem{definicija}{Дефиниција}[section]

\theoremstyle{plain}
\newtheorem{teorema}{Теорема}[section]
\newtheorem{lema}{Лема}[section]

\begin{document}

\maketitle
\tableofcontents
\newpage

\section{Увод}

Историја рачунарства је прича о људској потреби за бржим и прецизнијим рачунањем.  
Од најранијих механичких уређаја до модерних суперрачунара, рачунари су прошли кроз више етапа развоја.  
\textbf{Основни циљ} сваке ере је био исти: да се убрза обрада података и смањи број грешака.

\section{Рани рачунари}

\subsection{Абакус и механичке машине}

\begin{definicija}
\textbf{Абакус} је један од првих алата за рачунање, који се користио још у антици.
\end{definicija}

У 17. веку појављују се механичке машине за рачуњање, као што је \textit{Pascalina} Блеза Паскала.

\subsection{Тјуринг и идеја рачунара}

\begin{teorema}
Тјурингова машина може симулирати било који алгоритам, што значи да је основни модел савременог рачунара.
\end{teorema}

\begin{lema}
Сваки алгоритам који се може описати као низ корака може се реализовати на Тјуринговој машини.
\end{lema}

\section{Рачунари 20. века}

\subsection{Фон Нојманова архитектура}

\begin{definicija}
\textbf{Фон Нојманова архитектура} је концепт рачунарског система у коме се програм и подаци чувају у истој меморији.
\end{definicija}

Фон Нојманова архитектура је била основа за већину рачунара током 20. века.

\subsection{Прва генерација рачунара}

\begin{itemize}
    \item Коришћени су електронски вакуумски тијуби.
    \item Велики су и скупи.
    \item Нису били поуздани.
\end{itemize}

\section{Математички пример}

Позната формула из теорије информација:

\[
E = mc^2
\]

Али, у рачунарству је важна и сложеност алгоритама, нпр.:

\[
T(n) = O(n \log n)
\]

\section{Табела: генерације рачунара}

\begin{table}[h!]
\centering
\begin{tabular}{l c c}
\toprule
\textbf{Генерација} & \textbf{Технологија} & \textbf{Карактеристике} \\
\midrule
1. генерација & вакуумски тијуби & велики и непоуздани \\
2. генерација & транзистори & мањи и поузданији \\
3. генерација & интегрисани кругови & већа брзина \\
4. генерација & микропроцесори & персонални рачунари \\
5. генерација & вештачка интелигенција & паметни системи \\
\bottomrule
\end{tabular}
\caption{Генерације рачунара}
\end{table}

\section{Листе}

\subsection{Непоредне листе}

\begin{itemize}
    \item Кључни изуми:
    \begin{itemize}
        \item Абакус
        \item Механичке рачунске машине
        \item Тјурингова машина
        \item Први електронски рачунари
    \end{itemize}
\end{itemize}

\subsection{Поређане листе}

\begin{enumerate}
    \item Рани рачунари
    \item Механичке машине
    \item Електронски рачунари
    \item Модерни рачунари
\end{enumerate}

\section{Слика}

\begin{wrapfigure}{l}{0.35\textwidth}
    \vspace{-10pt}
    \includegraphics[width=0.34\textwidth]{ENIAC.jpg}
    \caption{Еволуција рачунара}
    \vspace{-10pt}
\end{wrapfigure}

На слици је приказана еволуција рачунара од механичких уређаја до модерних рачунара.  
\textcolor{blue}{Прва генерација} је била велика и непоуздана, док \textcolor{red}{четврта генерација} доноси персоналне рачунаре.

\section{Закључак}

Историја рачунарства показује да је свака нова генерација рачунара донела веће могућности и брже обраде података.  
\textbf{Развој рачунара} није само технички процес, већ и друштвени — рачунари су променили начин рада, учења и комуникације.  
\textit{Будућност} рачунарства ће вероватно донети још интелигентније системе и нове технологије.

\end{document}
