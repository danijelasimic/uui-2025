\documentclass{report}
\usepackage{graphicx} 
\usepackage[T2A]{fontenc}
\usepackage[utf8]{inputenc}
\usepackage[serbianc]{babel}
\usepackage{amsmath, amsthm}
\usepackage[export]{adjustbox}

\title{Анализа алгоритма selection sort}
\author{Александар Спасић}
\date{Фебруар 2026.}

\theoremstyle{definition}

\newtheorem{teorema}{Теорема}[chapter]

\begin{document}
\pagenumbering{gobble}
\maketitle
\clearpage
\newpage
\tableofcontents
\clearpage
\newpage
\setcounter{page}{1}
\chapter{Принцип рада selection sort}
\section{Увод}
Алгоритам \textit{selection sort} је један од најосновнијих алгоритама сортирања. Базира се на следећем принципу: ако низ има више од једног елемента, замени почетни елемент са најмањим елементом низа и затим аналогно сортирај остатак низа (елементе иза почетног). У свакој итерацији се на своју позицију доводи следећи по величини елемент низа, тј. у $i$-тој итерацији се $i$-ти по величини елемент доводи на позицију $i$. Ово се може реализовати тако што се пронађе позиција $m$ најмањег елемента од позиције $i$ до краја низа и затим се размене елемент на позицији $i$ и елемент на позицији $m$. Алгоритам се зауставља када се претпоследњи по величини елемент доведе на претпоследњу позицију у низу. \\

\noindent Пример:
\begin{itemize}
    \item Прикажимо рад алгоритма на примеру сортирања низа 2 12 10 15 20.
    \begin{enumerate}
        \item \textbf{.2} 12 10 15 20, $i = 0, m=4$, \textit{размена елемента 2 и 2}. 
        \item 2 \textbf{.12} 10 15 20, $i = 1, m=2$, \textit{размена елемента 12 и 10}. 
        \item 2 10 \textbf{.12} 15 20, $i = 2, m=2$, \textit{размена елемента 10 и 12}.
    \end{enumerate}
\end{itemize}
Визуелни пример рада је приказан на слици \ref{fig:placeholder}.
\begin{figure}[]
    \centering
    \includegraphics[width=0.5\linewidth, left]{slike/Selection-sort-1.png}
    \caption{Приказ рада алгоритма selection sort.}
    \label{fig:placeholder}
\end{figure}
\subsection{Временска сложеност}
\begin{teorema}
    Временска сложеност selection sort алгоритма је
    \begin{equation}
        O(n)=n^2
    \end{equation}
\end{teorema}
Значи, овај алгоритам је врло неефикасан. У табели \ref{tab:placeholder} је приказана сложеност алгоритма selection sort у односу на друге алгоритме сортирања:
\begin{table}[th]
    \centering
    \begin{tabular}{|c|c|}
        \hline
        Алгоритам & Временска сложеност \\
        \hline 
        Selection sort & $O(n)=n^2$ \\
        Quick sort & $O(n)=n^2$ \\
        Bubble sort & $O(n)=n^2$ \\
        Heap sort & $O(n) = n\log n$ \\
        Merge sort & $O(n) = n\log n$ \\
        \hline
    \end{tabular}
    \caption{Приказ временских сложености различитих алгоритама.}
    \label{tab:placeholder}
\end{table}
\section{Закључак}
Ионако алгоритам сортирања selection sort je врло неефикасан, он је врло важан да почетницима покаже елементарне технике писања алгоритама.
\end{document}
