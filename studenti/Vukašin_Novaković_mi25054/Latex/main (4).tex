\documentclass[12pt,a4paper]{article}
\usepackage{graphicx}
\usepackage[utf8]{inputenc}
\usepackage[T2A]{fontenc}
\usepackage[serbian]{babel}
\usepackage{amsmath, amssymb, amsthm}
\usepackage{float}
\usepackage{array}
\usepackage{enumitem}
\addto\captionsserbian{%
  \renewcommand{\figurename}{Слика}
}
\title{\textbf{Примена рачунара и вештачке интелигенције у трговању криптовалутама}}
\author{Вукашин Новаковић}
\date{18. јануар 2026.}

\begin{document}

\maketitle
\newpage
\tableofcontents
\vspace{7cm}
\section{Увод}
У овом раду ћемо укратко проћи кроз тему вештачке интелигенције и машина у свету крипто валута и "трејдовања". \\
Први овакви аутоматизовани системи су се појавили већ седамдесетих година двадесетог века, они су наравно били коришћени на финансијским тржиштима јер у то време још увек није било ни помисли на модерно крипто трговање. Ови системи су искључиво анализирали основне сигнале на берзи и били су веома једноставни. Тек почетком двадесет првог века по први пут настаје крипто тржиште и са њим и прве машине за модерно трејдовање.
\section{Модерне машине за крипто трговање}
Са појавом прве крипто валуте, односно \textbf{Bitcoin-а}, тада су се машине углавном ослањале искључиво на изглед графика који представља однос цене и времена у неком одређеном временском периоду \ref{fig:graf}.
\begin{figure}[H]
    \centering
    \includegraphics[width=0.7\textwidth]{Graf.png} 
    \caption{Приказ изгледа тренутног стања на берзи.}
    \label{fig:graf} 
\end{figure}
Касније са појавом интерфејса за програмирање апликација се мења и начин функционисања машина за трговину, они су омогућавали прављење стратегије која је сложенија од простог праћења графика. \\
Неки од нових машина су:
\begin{enumerate}
  \item Геко
  \item Хасбот
  \item Зенбот
\end{enumerate}
У модерно доба се машине углавном ослањају на вештачку интелигенцију која не само да прати цене него и сва дешавања у свету која могу на неки начин утицати на промену на тржишту.
Први пут да је овај тип трговања постао популаран и ушао у јавност јесте девети септембар две хиљаде прве када је услед велике трагедије у Њујорку "пало" цело тржиште и у настојећих пар сати након трагедије су цене свих деоница пале на веома ниску цену. У тих неколико сати су неколицина деоничара јасно предвидела "пад" и продали све своје деонице и касније при паду их поново купили. Људи који су се на овај начин обогатили су касније били сматрани за крајње нехумане персоне због профитирања на великој катастрофи. \\
\begin{table}[H]
\centering
\begin{tabular}{|c|p{10cm}|}
\hline
\textbf{Година} & \textbf{Догађај} \\
\hline
1970–1980 & Почетак аутоматизованог трговања на берзи акција \\
\hline
2009 & Пojava Bitcoin-а и први rule-based крипто ботови \\
\hline
2013 & API-ји омогућавају директно повезивање са платформама \\
\hline
2017–данас & AI и ML ботови, анализирају велике податке и предвиђају трендове \\
\hline
\end{tabular}
\caption{Кључни моменти развоја trading botova}
\end{table}
\section{Врсте машина}
Постоје четри типа машина које се данас на свакодневној бази користе широм света и они су:
\begin{itemize}
    \item Правилно базирани ботови
    \item Арбитражни ботови
    \item Тренд-пратећи ботови
    \item АИ/МЛ ботови
\end{itemize}
\subsection{Правилно базирани ботови}
Ови ботови прате једноставна правила где ако цена пређе преко одређене цене они купују а кад падне испод те ниже одређене цене продају.
\subsection{Арбитражни ботови}
Ови ботови су већ сложенији и прате цену исте валуте на више берзи у исто време. При томе користе једноставну формулу:
\[
P = (C_{\text{берза A}} - C_{\text{берза B}})\cdot V - F
\]
\begin{itemize}
    \item $P$ – остварени профит,
    \item $C_{\text{берза A}}$ – цена криптовалуте на берзи A,
    \item $C_{\text{берза B}}$ – цена криптовалуте на берзи B,
    \item $V$ – обим (количина) трговања,
    \item $F$ – укупне трансакционе накнаде.
\end{itemize}
\subsection{Тренд-пратећи ботови}
За разлику од осталих, овај бот се служи историјом цена ове валуте и користи је како би сигурно израчунао најбоље предвиђену цену помоћу следеће формуле:
\[
y = \beta_0 + \beta_1 x_1 + \dots + \beta_n x_n + \varepsilon
\]
\begin{itemize}
    \item $y$ – предвиђена вредност (нпр. будућа цена криптовалуте),
    \item $\beta_0$ – слободни члан (константа модела),
    \item $\beta_i$ – коефицијенти који одређују утицај појединачних променљивих,
    \item $x_i$ – улазне променљиве (цене, обим трговања, индикатори),
    \item $n$ – број улазних променљивих,
    \item $\varepsilon$ – грешка модела.
\end{itemize}
\subsection{АИ/МЛ ботови}
Ово су најразвијенији ботови који користе све могуће информације на које могу да наиђу и због тога успеју да дешифрују компликоване обрасце. Због тога се могу сматрати и најпрецизнијим од представљених.
\section{Закључак}
Аутоматизовано трговање криптовалутама представља значајну примену математике и информатике у савременим финансијским системима. Употребом статистичких модела, као што су трејдинг ботови омогућавају брзу анализу тржишта и доношење одлука у реалном времену. Иако ови системи могу повећати ефикасност и смањити људску грешку, неопходно је пажљиво управљање ризиком и разумевање ограничења математичких модела. Даљи развој вештачке интелигенције и машинског учења очекује се да ће имати све већу улогу у оптимизацији аутоматизованог трговања.
\end{document}
