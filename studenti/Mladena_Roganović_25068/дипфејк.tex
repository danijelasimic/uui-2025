\documentclass[12pt,a4paper]{article}

\usepackage[T2A]{fontenc}
\usepackage[utf8]{inputenc}
\usepackage[serbian]{babel}

\addto\captionsserbian{\renewcommand{\contentsname}{САДРЖАЈ}}
\addto\captionsserbian{\renewcommand{\abstractname}{САЖЕТАК}}
\addto\captionsserbian{\renewcommand{\refname}{ЛИТЕРАТУРА}}


\usepackage{graphicx}
\usepackage{color, xcolor}
\usepackage{amsthm}
\usepackage{url}
\usepackage{float}
\usepackage{microtype}
\usepackage{amsthm}

\theoremstyle{definition}
\newtheorem{дефиниција}{Дефиниција}

\usepackage[unicode]{hyperref}
\hypersetup{colorlinks,citecolor=red,filecolor=red,linkcolor=blue,urlcolor=blue}

\begin{document}

\begin{titlepage}
\newcommand{\HRule}{\rule{\linewidth}{0.4mm}}

\center 
	
\textsc{\LARGE Математички факултет}\\[3cm] 
	
\textsc{\Large Семинарски рад}\\[0.1cm]
	
\textsc{\large из увода у информатику}\\[0.5cm] 
	
\HRule\\[0.4cm]
	
{\LARGE\bfseries Етика и ризици дипфејк технологије}\\[0.2cm] 
	
\HRule\\[2cm]

\vspace{17\baselineskip}
\begin{minipage}{0.4\textwidth}
		\begin{flushleft}
			\large
			\textit{Студент}\\
			Младена Рогановић, 68/2025
		\end{flushleft}
	\end{minipage}
\hspace*{1cm}
	\begin{minipage}{0.4\textwidth}
		\begin{flushright}
			\large
			\textit{Професор}\\
			Данијела Симић
		\end{flushright}
	\end{minipage}
	
	
\vfill\vfill\vfill\vfill 
	
{\large Beograd, \today} 
	
\vfill 
	
\end{titlepage}

\tableofcontents

\newpage
\abstract {У раду се разматра дипфаке технологија као савремени производ развоја вештачке интелигенције, са посебним освртом на принципе њеног рада, области примене и етичке, друштвене и правне изазове које она носи. Објашњен је начин функционисања генеративних супарничких мрежа и приказане су могућности употребе деепфаке технологије у образовању, медицини, уметности и филмској\\ индустрији, где она може имати значајне позитивне ефекте. \\Истовремено, анализирани су и ризици који проистичу из њене злоупотребе, попут повреде приватности, нарушавања угледа,\\ манипулације информацијама и угрожавања поверења у дигиталне медије. Посебна пажња посвећена је етичким дилемама и недовољној усаглашености постојећих правних оквира брзом развоју ове\\ технологије. Циљ рада је да укаже на потребу за одговорном применом дипфаке технологије, развојем друштвене свести, као и усавршавањем етичких смерница и законске регулативе, како би се њени позитивни потенцијали искористили, а негативне последице свеле на минимум.}

\section{Увод}

Развој \textbf{вештачке интелигенције} довео је до појаве технологија које значајно утичу на начин на који се ствара и перципира дигитални садржај. Једна од најпознатијих и најконтроверзнијих технологија тог типа јесте \textit{дипфејк}.

Ова технологија омогућава креирање изузетно уверљивих, али лажних аудио и видео записа, што отвара бројна \textcolor{red}{етичка}, \textcolor{orange}{правна} и \textcolor{blue}{друштвена} питања.

\section{Основе дипфејк технологије}

\subsection{Начин функционисања}

Дипфејк се најчешће заснива на примени генеративних супарничких мрежа (GAN), које се састоје од два модела:
\begin{itemize}
    \item \textbf{генератора}, који ствара лажне садржаје,
    \item \textbf{дискриминатора}, који покушава да их препозна \cite{1,5}.
\end{itemize}

\begin{дефиниција}
Дипфејк технологија представља примену алгоритама машинског учења ради креирања синтетичких медијских садржаја који имитирају стварне људе.
\end{дефиниција}

\section{Примене технологије}

\subsection{Позитивни аспекти}

Предности дипфејк технологије могу се сагледати кроз следеће области:
\begin{enumerate}
    \item образиявње и симулације
    \item уметност и дигитална креативност
    \item медицина и професионални тренинг
\end{enumerate}

\subsection{Пример поређења}

\begin{center}
\begin{tabular}{|c|c|}
\hline
\textbf{Област} & \textbf{Начин примене} \\
\hline
Образовање & Интерактивне симулације \\
Медији & Дигитална продукција \\
Медицина & Обука медицинског кадра \\
\hline
\end{tabular}
\end{center}

\section{Изазови и етичка питања}

Иако доноси бројне могућности, дипфејк технологија носи и одређене проблеме:
\begin{itemize}
    \item злоупотреба идентитета
    \item нарушавање приватности
    \item губитак поверења у медије
\end{itemize}

Посебно је важно истаћи да \textbf{неодговорна употреба} може имати дугорочне последице по друштво\cite{6}.

\section{Закључак}

Дипфејк технологија представља моћан алат савременог дигиталног доба. Иако омогућава иновације у различитим областима, неопходно је њено одговорно коришћење, уз развој јасних етичких и правних оквира, како би се спречиле потенцијалне злоупотребе.
\newpage
\section{Додатак}
У овом поглављу сам додала пример неке математичке формуле, како би се допунио формат тражен за документ.\\
\begin{center}
    $ \displaystyle \sum_{i=1}^{n} \pi i^2 $
\end{center}

\addcontentsline{toc}{section}{Литература}
\appendix
\iffalse
\bibliography{семинарски} 
\bibliographystyle{plain}
\fi

\begin{thebibliography}{9}

\bibitem{1} I. Kalpokas and J. Kalpokiene, \emph{Deepfakes: A Realistic Assessment of Potentials, Risks and Policy Regulation}. Springer, Cham, 2022.

 \bibitem{2} S. Taneja, S. Gupta, M. Kukreti, and A. S. Chauhan, Eds., \emph{Mastering Deepfake Technology: Strategies for Ethical Management and Security}. River Publishers, 2025.

 \bibitem{3} A. Raina and G. Mann, “Exploring the Ethics of Deepfake Technology in Media: Implications for Trust and Information Integrity,” \emph{J. Interdiscip. Ethics Res.}, vol. 4, no. 3, 2024.

 \bibitem{4} S. L. Burton and D. P. Harvie, “Deepfakes: Unmasking the Technological, Societal, and Ethical Dimensions,” \emph{Journal of Interdisciplinary Ethics and Research}, vol. 9, no. 2, 2025.

 \bibitem{5} D. K. Citron and R. Chesney, “The Distinct Wrong of Deepfakes,” \emph{Philosophy \& Technology}, vol. 34, pp. 1311–1332, 2021.

 \bibitem{6} I. Čučilović, “Deepfake tehnologija – krivičnopravne implikacije,” \emph{Crimen: Časopis za krivične nauke}, 2024.
 
\end{thebibliography}



\end{document}
