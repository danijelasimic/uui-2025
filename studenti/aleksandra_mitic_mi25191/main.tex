\documentclass[12pt,a4paper]{article}

\usepackage[T2A]{fontenc}
\usepackage[utf8]{inputenc}
\usepackage[serbian]{babel}
\usepackage{graphicx}
\usepackage{float}
\usepackage{wrapfig}
\usepackage{amsmath, amsthm, amssymb}
\usepackage{xcolor}
\usepackage{hyperref}
\usepackage{geometry}
\usepackage[serbian]{babel}
\geometry{margin=2.5cm}

\title{\textbf{Историја тркачких аутомобила и савремени рачунарски системи}}
\author{Александра Митић}
\date{17.2.2026.}

\addto\captionsserbian{\renewcommand{\contentsname}{Садржај}}
\addto\captionsserbian{\renewcommand{\refname}{Литература}}

\newtheorem{definicija}{Дефиниција}
\newtheorem{teorema}{Теорема}
\newtheorem{lema}{Лема}
\begin{document}

\maketitle
\tableofcontents
\newpage

\section{Увод}

Историја тркачких аутомобила представља спој инжењерства, физике и рачунарских наука. 
Од првих механичких возила крајем XIX века до модерних болида у \textbf{Формули 1} 
\cite{auto_history}, развој технологије је омогућио значајан напредак у брзини и безбедности.

\section{Историјски развој тркачких аутомобила}

\subsection{Почеци ауто-трка}

Прве организоване трке одржане су крајем XIX века у Француској \cite{auto_history}. 
Рани модели су били потпуно механички, без електронских система.

\begin{wrapfigure}{l}{0.45\textwidth}
    \centering
    \includegraphics[width=0.43\textwidth]{early-sport-car-picture.jpg}
    \caption{Рани модел тркачког аутомобила}
\end{wrapfigure}

Развој мотора са унутрашњим сагоревањем омогућио је повећање снаге. 
Основна формула снаге мотора може се приказати као:

\[
P = F \cdot v
\]

где је $P$ снага, $F$ сила, а $v$ брзина \cite{f1tech}.

\subsection{Ера аеродинамике}

Средином XX века почиње примена аеродинамичких принципа \cite{f1tech}. 
Отпор ваздуха израчунава се формулом:

\[
F_d = \frac{1}{2} \rho v^2 C_d A
\]

где је $\rho$ густина ваздуха, $C_d$ коефицијент отпора, а $A$ површина.

\section{Савремени рачунарски системи у аутомобилима}

\subsection{Хардвер у модерним болидима}

Данашњи тркачки аутомобили садрже више од 100 сензора и сложене електронске контролне јединице (ECU) \cite{embedded}. 
\textit{Микропроцесори} анализирају податке у реалном времену \cite{embedded}.

\textcolor{blue}{Савремени системи омогућавају оптимизацију перформанси током саме трке \cite{f1tech}.}

\begin{definicija}
Електронска контролна јединица (ECU) је уграђени систем који управља радом мотора и прикупља податке са сензора.
\end{definicija}

\begin{lema}
Ако се повећа прецизност мерења сензора, повећава се и ефикасност управљања мотором \cite{embedded}.
\end{lema}

\begin{teorema}
Интеграција рачунарских система у тркачке аутомобиле доводи до статистички значајног побољшања перформанси у односу на чисто механичке системе \cite{f1tech}.
\end{teorema}

\subsection{Типови система}

Постоје различити типови система:

\begin{itemize}
    \item Системи за контролу мотора
    \item Телеметријски системи\footnote{Прикупљају податке са сензора у реалном времену и шаљу их инжењерском тиму.}
    \item Безбедносни системи\footnote{Системи који обезбеђују сигурност возача и стабилност возила, нпр. ABS и системи за спречавање клизања.}
\end{itemize}


Нумерисана листа развојних фаза:

\begin{enumerate}
    \item Механички период
    \item Електромеханички период
    \item Дигитални период
\end{enumerate}

\section{Поређење технологија}

\begin{table}[H]
\centering
\begin{tabular}{|c|c|c|}
\hline
Период & Тип система & Максимална брзина \\ \hline
1900 & Механички & 80 km/h \\ \hline
1950 & Аеродинамички & 250 km/h \\ \hline
2024 & Дигитални + ECU & 370 km/h \\ \hline
\end{tabular}
\caption{Развој технологије кроз време \cite{auto_history}}
\end{table}

\section{Закључак}

Развој тркачких аутомобила показује како инжењерство и рачунарство заједно обликују будућност транспорта. 
Од једноставних механичких конструкција до сложених дигиталних система, 
евидентно је да је \textbf{рачунарска технологија} постала кључни фактор успеха \cite{embedded}.

\newpage
\begin{thebibliography}{9}

\bibitem{f1tech}
Carroll Smith (1978). \textit{Tune to Win: The art and science of race car development and tuning}. Aero Publishers, Inc.

\bibitem{auto_history}
Karl Ludvigsen (2018). \textit{Classic Grand Prix Cars: The Glorious Prehistory of Formula 1: 1906-1960}. Bentley Publishers.

\bibitem{embedded}
M. Kathiresh (2022) \textit{Automotive Embedded Systems: Key Technologies, Innovations, and Applications (EAI/Springer Innovations in Communication and Computing)}.

\end{thebibliography}

\end{document}
