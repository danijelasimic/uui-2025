%! Author = Nikola Marinkovic
%! Date = 13/04/2006

\documentclass[a4paper]{article}

\usepackage[T2A]{fontenc}
\usepackage{noto}
\usepackage[utf8]{inputenc}

\usepackage[serbianc]{babel}
\usepackage[unicode]{hyperref}

\usepackage{graphicx}
\usepackage{xcolor}

\newtheorem{definicija}{Дефиниција}
\newtheorem{teorema}{Теорема}
\newtheorem{lema}{Лема}

\newcommand{\eng}[1]{
    (\textit{енг.} #1)
}

\begin{document}

    \begin{titlepage}
        \newcommand{\HRule}{\rule{\linewidth}{0.4mm}}

        \center

        \textsc{\LARGE Математички факултет}\\[3cm]

        \textsc{\Large Семинарски рад}\\[0.1cm]

        \textsc{\large из увода у информатику}\\[0.5cm]

        \HRule\\[0.4cm]

        {\LARGE\bfseries Алгоритам сортирања уметањем}\\[0.2cm]

        \HRule\\[2cm]

        \vspace{17\baselineskip}



        \begin{minipage}{0.4\textwidth}
            \begin{flushleft}
                \large
                \textit{Студент}\\
                Никола Маринковић 52/2025
            \end{flushleft}
        \end{minipage}
        \hspace*{1cm}
        \begin{minipage}{0.4\textwidth}
            \begin{flushright}
                \large
                \textit{Професор}\\
                др.\ Данијела Симић
            \end{flushright}
        \end{minipage}


        \vfill\vfill\vfill\vfill

        {\large Београд, \today}

        \vfill

    \end{titlepage}

    \tableofcontents

    \newpage

    \section{Увод}
    \label{sec:uvod}

    Одвајкада, једна од насушних потреба свакога ко се бавио рачунарским наукама било је сортирање података.
    Тежња ка овом циљу изродила је многе алгоритме, сваки са сопственим предностима и манама.
    Један од таквих алгоритама је алгоритам сортирања уметањем, о коме ће бити речи у овом раду.

    \section{Основни појмови}
    \label{sec:osn_pojmovi}

    \begin{definicija}
        Алгоритам сортирања уметањем је алгоритам који пролази кроз низ и сваки елемент убацује на исправну позицију у већ сортираном делу низа.
    \end{definicija}

    Алгоритам је \color{red}једноставан \color{black} и \color{red}интуитиван\color{black}, и због тога се често користи у образовне сврхе.
    Његова основна идеја је слична начину на који човек сортира карте у руци: један по један елемент се убацује на одговарајуће место у већ сортирани део низа.

    \subsection{Основне карактеристике}
    \label{subsec:osn_karakt}

    \begin{itemize}
        \item Једноставан за имплементацију
        \item Стабилан алгоритам
        \item Користи константну количину меморије (\textit{in-place})
    \end{itemize}

    \subsection{Примене}
    \label{subsec:primene}

    \begin{enumerate}
        \item Мали скупови података
        \item Делимично сортирани низови
        \item Компонента сложенијих алгоритама
    \end{enumerate}


    \section{Временска сложеност}
    \label{sec:time_complexity}

    Временска сложеност алгоритма у најгорем случају дата је формулом:

    \begin{equation}
        \label{eq:vr_sl}
        T(n) = \sum_{i=1}^{n} i = \frac{n(n+1)}{2}
    \end{equation}

    Позната физичка формула, која овде нема директну примену али служи као пример, гласи:

    \begin{equation}
        \label{eq:ajnstajn}
        E = mc^2
    \end{equation}

    \begin{lema}
        У најгорем случају, алгоритам сортирања уметањем има квадратну временску сложеност.
    \end{lema}



    \section{Поређење сложености}
    \label{sec:poredjenje}

    \begin{table}[h]
        \label{table:poredjenje_slozenosti}
        \centering
        \begin{tabular}{|c|c|c|}
            \hline
            Алгоритам & Најбољи случај & Најгори случај \\
            \hline
            Insertion Sort & $O(n)$ & $O(n^2)$ \\
            Bubble Sort & $O(n)$ & $O(n^2)$ \\
            Selection Sort & $O(n^2)$ & $O(n^2)$ \\
            \hline
        \end{tabular}
        \caption{Поређење алгоритама сортирања}
    \end{table}



    \section{Илустрација алгоритма}
    \label{sec:ilustracija}

    \begin{figure}[h]
        \label{figure:slika}
        \begin{minipage}{0.4\textwidth}
            \includegraphics[width=\linewidth]{./slike/insertion_sort.jpg}
        \end{minipage}
        \hfill
        \begin{minipage}{0.55\textwidth}
            Алгоритам ради тако што постепено гради сортирани део низа.
            \textit{Сваком новом елементу} се тражи одговарајуће место.
        \end{minipage}
        \caption{Визуелни приказ Insertion Sort алгоритма}
    \end{figure}


    \section{Формална анализа}
    \label{sec:formalna_analiza}

    \begin{teorema}
        Ако је улазни низ већ сортиран, алгоритам сортирања уметањем ради у линеарном времену.
        У том случају, сваки елемент је већ на исправној позицији, па није потребно померање.
    \end{teorema}



    \section{Закључак}
    \label{sec:zakljucak}

    Алгоритам сортирања уметањем је \textbf{једноставан, поучан и користан} у одређеним случајевима.
    Иако није погодан за велике скупове података, у најбољем случају његова ефикасност и лакоћа имплементације чине га важним алгоритмом у образовању.


\end{document}