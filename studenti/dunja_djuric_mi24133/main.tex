\documentclass[14pt]{article}

\usepackage[T2A]{fontenc}
\usepackage[utf8]{inputenc}
\usepackage[english,serbianc]{babel}
\usepackage{amsmath,amsthm}
\usepackage{xcolor}
\usepackage{graphicx}
\usepackage{geometry}
\geometry{margin=2.3cm}

\definecolor{dgreen}{RGB}{24, 123, 51}
\definecolor{yellow2}{RGB}{136, 123, 51} 
\definecolor{ourple}{RGB}{126, 88, 187}
\definecolor{lblue}{RGB}{0, 35, 106}

\title{\textbf{Основни појмови фракталне геометрије}}
\author{Дуња Ђурић}
\date{18. фебруар 2026.}

\newtheorem{definicija}{Дефиниција}
\newtheorem{lema}{Лема}
\newtheorem{teorema}{Теорема}

\begin{document}
\maketitle
\tableofcontents

\section{Увод}

\textbf{\color{lblue}Фрактали} представљају важну област савремене математике која проучава објекте са сложеном структуром. Посебно су интересантни због своје самосличности, односно појаве истих образаца на различитим нивоима увећања. Данас се фрактали примењују у бројним научним дисциплинама, укључујући физику, биологију и рачунарску графику.

\begin{figure}[h]
\begin{center}
\includegraphics[scale=0.3]{slike/mandelbrotov_skup.jpg}
\end{center}
\caption{Пример фракталне структуре}
\end{figure}

\section{Основни појмови}

\begin{definicija}
\textbf{\color{lblue}Фрактали} се најчешће добијају итеративним поступцима који понављају једноставна правила. 
\end{definicija}
Иако су формуле једноставне, резултујуће структуре могу бити изузетно сложене. Ова особина чини \textbf{\color{lblue}фрактале} погодним моделима природних појава.

\subsection{Особине}

\textbf{\color{lblue}Фрактали} се одликују низом специфичних карактеристика које их издвајају од класичних геометријских објеката. Најважније су:
\begin{enumerate}
\item самосличност
\item бесконачни детаљи
\item нелинеарна структура
\end{enumerate}

\begin{lema}
Увећањем \textbf{\color{lblue}фрактала} увек се појављују нови обрасци.
\end{lema}


\begin{teorema}
\textbf{\color{lblue}Фрактали} имају своју фракталну димензију, која је већа од тополошке и често је разломак. 
\end{teorema}
Суштина ове теореме је да мери како се или садржај објекта мења када промените скалу мерења. Ова зависност показује да сложеност објекта не расте линеарно са променом скале, већ по степеном закону који одражава његову самосличност.

{\color{yellow2}Фрактална димензија $(D)$} се израчунава помоћу релације између {\color{ourple}броја нових делова по итерацији $(N)$} и {\color{dgreen}фактора скалирања $(S)$}, и дата је формулом:

\[
{\color{yellow2}D}=\frac{\log({\color{ourple}N})}{\log({\color{dgreen}S})}
\]

Најпознатији \textbf{\color{lblue}фрактали} и њихове особине:

\begin{center}
\begin{tabular}{|l|c|c|c|p{4.5cm}|}
\hline
\textbf{Назив \color{lblue}фрактала} & \textbf{$N$} & \textbf{$S$} & \textbf{$D$} & \textbf{Опис структуре} \\
\hline
Канторов скуп & 2 & 3 & $\approx 0.6309$ & Настаје уклањањем средње трећине дужи. \\
\hline
Кохова крива & 4 & 3 & $\approx 1.2618$ & Бесконачно дуга линија која не пресеца саму себе. \\
\hline
Троугао Сјерпинског & 3 & 2 & $\approx 1.5850$ & Троугао састављен од три мања троугла. \\
\hline
Тепих Сјерпинског & 8 & 3 & $\approx 1.8928$ & Квадрат са бесконачно много рупа, површина му тежи нули. \\
\hline
Менгеров сунђер & 20 & 3 & $\approx 2.7268$ & Тродимензионална верзија тепиха; бесконачна површина и нулта запремина. \\
\hline
\end{tabular}
\end{center}

\section{Примене \textbf{\color{lblue}фрактала}}
\textbf{\color{lblue}Фрактали} се примењују у:
\begin{itemize}
\item компјутерској графици, где се \textbf{\color{lblue}фрактали} користе за генерисање сложених визуелних структура и природних пејзажа,
\item медицини за анализу структуре ткива,
\item телекомуникацијама за дизајн компактних антена које примају широк спектар фреквенција, анализу сигнала и обраду података са сложеном динамиком,
\item моделирању природних појава, као што су обале, облаци и биљке, итд.
\end{itemize}

\section{Закључак}

Фрактална геометрија показује како једноставни алгоритми могу довести до изузетно сложених структура значајних за савремену науку.

\end{document}

