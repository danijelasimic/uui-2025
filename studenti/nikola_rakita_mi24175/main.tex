\documentclass[a4paper]{article}


\usepackage{color}
\usepackage{url}
\usepackage[T2A]{fontenc}
\usepackage[utf8]{inputenc}
\usepackage{graphicx}
\usepackage[english,serbian]{babel}
\usepackage[unicode]{hyperref}
\usepackage[serbian]{babel}

\addto\captionsserbian{%
  \renewcommand{\contentsname}{Садржај}%
  \renewcommand{\refname}{Литература}%
  \renewcommand{\abstractname}{Сажетак}%
}


\hypersetup{colorlinks,citecolor=green,filecolor=green,linkcolor=blue,urlcolor=blue}


\newtheorem{definicija}{Дефиниција}[section]
\newtheorem{lema}{Лема}[section]
\newtheorem{teorema}{Теорема}[section]


\begin{document}


\begin{titlepage}


\newcommand{\HRule}{\rule{\linewidth}{0.4mm}}


\center


\textsc{\LARGE Математички факултет}\\[3cm]


\textsc{\Large Семинарски рад}\\[0.1cm]


\textsc{\large из увода у информатику}\\[0.5cm]


\HRule\\[0.4cm]


{\LARGE\bfseries Због чега сендвичи увек падају на намазану страну}\\[0.2cm]


\HRule\\[2cm]


\vspace{15\baselineskip}


\begin{minipage}{0.4\textwidth}
\begin{flushleft}
\large
\textit{Студент}\\
Никола Ракита 175/24
\end{flushleft}
\end{minipage}
\hspace*{1cm}
\begin{minipage}{0.4\textwidth}
\begin{flushright}
\large
\textit{Професор}\\
Данијела Симић
\end{flushright}
\end{minipage}


\vfill


{\large Београд, 19. јануар 2026.}


\end{titlepage}


\tableofcontents
\newpage


\abstract{
Феномен падања сендвича на намазану страну често се у свакодневном животу приписује Марфијевом закону. Ипак, ова појава има јасно и проверљиво научно објашњење засновано на законима класичне механике. У раду се анализира утицај гравитације, геометрије и ротационог кретања на исход пада сендвича, уз коришћење једноставних математичких модела и експерименталних посматрања.
}


\section{Увод}
Фраза да сендвич увек пада на намазану страну дубоко је укорењена у популарној култури. На први поглед, ова појава делује као случајност или пример Марфијевог закона. Међутим, физика пружа конкретно објашњење засновано на законима кретања и ротације крутих тела.


Циљ овог рада је да покаже да исход пада сендвича није последица лоше среће, већ предвидив физички процес.


\section{Марфијев закон и научни приступ}
Марфијев закон се често користи као хумористично објашњење неповољних исхода. Ипак, у случају сендвича, анализа показује да постоје објективни разлози зашто је вероватније да намазана страна заврши окренута ка поду.


\begin{itemize}
\item интуитивно објашњење – Марфијев закон,
\item научно објашњење – класична механика,
\item образовни значај феномена.
\end{itemize}


\section{Основни појмови}


\begin{definicija}
Сендвич је приближно круто тело које након губитка ослонца врши транслационо и ротационо кретање под утицајем гравитације.
\end{definicija}


\section{Физичко и математичко објашњење}
Када сендвич клизне са стола, тежиште се налази изван ивице ослонца, што изазива момент силе и почетак ротације. Због релативно мале висине стола, сендвич нема довољно времена да изврши пуну ротацију.


\subsection{Математички модел}
У поједностављеном моделу, време пада $t$ са висине $h$ дато је формулом:
\[
t = \sqrt{\frac{2h}{g}}
\]
где је $g$ гравитационо убрзање. За типичне висине столова, угао ротације остаје мањи од $\pi$, што доводи до пада на намазану страну.


\begin{lema}
За висине мање од једног метра, сендвич не може извршити пуну ротацију током пада.
\end{lema}


\begin{teorema}
При типичним кухињским условима, вероватноћа да сендвич падне на намазану страну већа је од 50\%.
\end{teorema}


\section{Слика и табела}


На слици \ref{fig:sendvic} приказан је шематски приказ ротације сендвича током пада.


\begin{figure}[h!]
\centering
\includegraphics[width=0.5\textwidth]{sendvic.png}
\caption{Ротација сендвича током пада}
\label{fig:sendvic}
\end{figure}


У табели \ref{tab:visina} приказан је утицај висине стола на исход пада.


\begin{table}[h!]
\centering
\caption{Утицај висине на исход пада}
\begin{tabular}{|c|c|c|} \hline
Висина (m) & Ротација & Исход \\ \hline
0.5 & 1/4 круга & Намаз доле \\ \hline
0.8 & 1/2 круга & Намаз доле \\ \hline
1.5 & 1 круг & Насумично \\ \hline
\end{tabular}
\label{tab:visina}
\end{table}


\section{Закључак}
Иако се често верује да сендвич пада на намазану страну због Марфијевог закона, анализа показује да је овај феномен последица физичких закона. Комбинација гравитације, геометрије и ограничене висине стола довољна је да се објасни ова свакодневна појава.


\addcontentsline{toc}{section}{Литература}
\begin{thebibliography}{9}


\bibitem{murphy} E. Murphy. \emph{Murphy's Law}. American Engineering Journal, 1949.


\bibitem{physics} R. Cross. \emph{Why buttered toast falls buttered-side down}. European Journal of Physics, 2001.


\bibitem{mechanics} H. Goldstein. \emph{Classical Mechanics}. Addison-Wesley, 1980.


\end{thebibliography}
\end{document}