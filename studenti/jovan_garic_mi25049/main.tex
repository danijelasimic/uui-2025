\documentclass[a4paper]{article}

\usepackage{url}
\usepackage[T2A]{fontenc}
\usepackage[utf8]{inputenc}
\usepackage{graphicx}
\usepackage{wrapfig}
\usepackage[english,serbianc]{babel}

\usepackage[unicode]{hyperref}
\hypersetup{colorlinks=true,linkcolor=black,urlcolor=black,citecolor=black}

\usepackage{amsmath,amsthm,amssymb}
\usepackage{float}

\title{\textbf{Мала Фермаова теорема}}
\author{Јован Гарић}
\date{\today}

% Окружења
\newtheorem{definicija}{Дефиниција}
\newtheorem{teorema}{Теорема}
\newtheorem{lema}{Лема}

\begin{document}

\maketitle

\newpage

\tableofcontents

\newpage

\section{Увод}

Мала Фермаова теорема је део математичке области теорије бројева. У наставку рада говорићемо више о њој и њеном значају, као и математичару који ју је открио и по коме носи име.

\newpage

\section{Мала Фермаова теорема}

\begin{wrapfigure}{r}{0.35\textwidth}
    \centering
    \includegraphics[width=0.33\textwidth]{fermat.jpg}
    \caption{Пјер де Ферма}
\end{wrapfigure}

Малу Фермаову теорему формулисао је француски математичар \textbf{Пјер де Ферма} у 17. веку, у оквиру својих истраживања о својствима простих бројева. Иако је теорему изнео без доказа, она је касније потврђена од стране више математичара и представља темељ многих савремених алгоритама.

Први математичар који ју је доказао био је \textbf{Готфрид Лајбниц} и његов доказ је пронађен у рукопису без датума.

Пре него што прикажемо теорему, прво ћемо дефинисати појмове конгруенције и простих бројева, на којима се ова теорема заснива.

\subsection{Конгруенције}

\begin{definicija}
Нека су $a,b,m \in \mathbb{Z}$, $m>0$. Кажемо да су $a$ и $b$
\textbf{конгруентни по модулу $m$} ако важи
\[
a \equiv b \pmod{m}.
\]
Ово значи да бројеви $a$ и $b$ дају исти остатак при дељењу бројем $m$.
\end{definicija}

\begin{table}[H]
\centering
\begin{tabular}{|c|c|c|}
\hline
$a$ & $m$ & $a \bmod m$ \\
\hline
10 & 3 & 1 \\
25 & 7 & 4 \\
47 & 5 & 2 \\
\hline
\end{tabular}
\caption{Примери остатака при дељењу.}
\end{table}

\subsection{Прости бројеви}

Природан број $p > 1$ је \textbf{прост} ако је дељив само {\emсамим собом} и {\em бројем $1$}. Број $1$ \textbf{није} ни прост, ни сложен.

\begin{teorema}[Еуклид]
Постоји бесконачно много простих бројева.
\end{teorema}

\begin{lema}
Ако прост број $p$ дели производ $ab$, онда важи да $p$ дели $a$ или $p$ дели $b$.
\end{lema}

\subsection{Теорема}

Ако је $p$ {\em прост број} и \textbf{не дели} број $a$, онда је:

\[
a^{p-1} \equiv 1 \pmod{p}
\]

Ова теорема има кључну примену у \textbf{криптографији}.

\newpage

\section{Примене}

Неке од примена ове теореме су:

\begin{itemize}
    \item \textbf{Криптографија(RSA и други алгоритми)}
    \item \textbf{Убрзање модуларног степеновања}
    \item \textbf{Фермаов тест простости}
\end{itemize}

\newpage

\section{Закључак}

У овом раду приказане су конгруенције и прости бројеви, мала Фермаова теорема, као и неке од примена у математичким и рачунарским областима.

\end{document}