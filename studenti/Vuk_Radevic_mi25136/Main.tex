\documentclass[12pt,a4paper]{article}

\usepackage[utf8]{inputenc}
\usepackage[T2A]{fontenc}
\usepackage[serbian]{babel}

\usepackage{amsmath, amssymb, amsthm}

\usepackage{graphicx}
\usepackage{float}
\usepackage{wrapfig}
\usepackage{booktabs}
\usepackage{mwe}
\usepackage{xcolor}
\usepackage{geometry}
\geometry{margin=2.5cm}
\usepackage[colorlinks=true, linkcolor=blue]{hyperref}

\newtheorem{teorema}{Теорема}
\newtheorem{lema}{Лема}
\theoremstyle{definition}
\newtheorem{definicija}{Дефиниција}

\title{\textbf{Биномна теорема}}
\author{Вук Радевић}
\date{\today}

\begin{document}
	
	\maketitle
	
	\newpage
	\tableofcontents
	\newpage
	
	\section{Увод}
	
	\textbf{Биномна теорема} представља један од основних резултата алгебре.
	Она описује начин развијања степена збира два члана облика $(a+b)^n$.
	У овом раду биће приказана формула биномне теореме, њено математичко значење и практична примена.\\
	Има широку примену у математици, статистици, теорији вероватноће, па чак и у физици и информатици. Дубоко је повезана са \textbf{Паскаловим троуглом}.
	
	\section{Основне информације}
	\subsection{Опште}
	
	\begin{definicija}
		Бином је алгебарски израз који се састоји од два члана, на пример $a+b$.
	\end{definicija}
	
	Биномни коефицијенти дефинисани су формулом:
	
	\[
	\binom{n}{k} = \frac{n!}{k!(n-k)!}
	\]
	
	\begin{lema}
		Збир свих биномних коефицијената за фиксно $n$ једнак је $2^n$.
	\end{lema}
	
	\begin{teorema}
		За сваки природан број $n$ важи:
		\[
		(a+b)^n = \sum_{k=0}^{n} \binom{n}{k} a^{n-k} b^k
		\]
	\end{teorema}
	
	Где пример развоја кубног бинома изгледа овако:
	
	\[
	(a+b)^3 = a^3 + 3a^2b + 3ab^2 + b^3
	\]
	
	\subsection{Кораци рада}
	Праћењем следећих корака видимо како се решава биномна формула:
	\begin{enumerate}
		\item У запису формуле одређујемо вредност $n$ (\textcolor{purple}{на пример $n=2$})
		\item Израчунавамо коефицијенте - користимо формулу за њихово израчунавање
		\item Помножимо све са одговарајућим коефицијентом - Паскалов троугао
		\item Сабирамо све чланове
	\end{enumerate}
	
	\section{Уска повезаност са Паскаловим троуглом}
	\textbf{Паскалов троугао} је троугаона шема бројева у којој се сваки број добија као збир два броја изнад њега. Редови овог троугла представљају биномне коефицијенте, односно коефицијенте који се појављују у развоју израза облика $(a+b)^n$. \textcolor{purple}{На пример, четврти ред (1, 3, 3, 1) одговара коефицијентима у развоју $(a+b)^3$}. Због тога Паскалов троугао представља једноставан и прегледан начин за одређивање коефицијената у биномној теореми.
	
	\begin{table}[H]
		\centering
		\begin{tabular}{|c|cccc|}
			\hline
			$n$ & $k=0$ & $k=1$ & $k=2$ & $k=3$\\
			\hline
			1 & 1  &   &   & \\
			2 & 1 & 1 &   & \\
			3 & 1 & 2 & 1 & \\
			4 & 1 & 3 & 3 & 1 \\
			\hline
		\end{tabular}
		\caption{Део Паскаловог троугла}
	\end{table}
	
	\section{Примена бинарне теореме}
	
	\begin{wrapfigure}{l}{0.5\textwidth}
		\centering
		\includegraphics[width=0.45\textwidth]{Slike/Binomna formula.jpg}
		\caption{Биномна формула} 
	\end{wrapfigure}
	
	Постоји много примена,\\ али неке од главних су:
	
	\begin{itemize}
		\item У комбинаторици
		\item У теорији вероватноће 
		\item У алгебри за развијање полинома
		\item У анализи за приближна рачунања
	\end{itemize}
	
	\section{Закључак}
	Биномна формула могућава једноставно развијање степена збира и има широку примену
	у математици и природним наукама. Помоћу биномних коефицијената и Паскаловог троугла могуће је лако одредити све чланове развоја без дуготрајног множења.
	
\end{document}