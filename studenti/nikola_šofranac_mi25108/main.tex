\documentclass[12pt,a4paper]{article}

\usepackage{fontspec}
\usepackage[serbianc]{babel}

\setmainfont{Times New Roman}

\usepackage{geometry}
\usepackage{xcolor}
\usepackage{graphicx}
\usepackage{amsmath, amsthm}
\theoremstyle{definition}
\newtheorem{definicija}{Дефиниција}
\newtheorem{teorema}{Теорема}
\newtheorem{lema}{Лема}
\usepackage{booktabs}
\usepackage{caption}
\usepackage{float}
\usepackage{hyperref}
\usepackage{setspace}

\geometry{margin=2.5cm}
\onehalfspacing

\title{Зашто је пиратерија данас постала толико популарна}
\author{Никола Шофранац}
\date{2. јануар 2026.}

\begin{document}

\begin{titlepage}
\thispagestyle{empty}

\begin{center}

{\Large \textbf{УНИВЕРЗИТЕТ У БЕОГРАДУ}}\\[0.3cm]
{\Large \textbf{МАТЕМАТИЧКИ ФАКУЛТЕТ}}\\[2.5cm]

{\large Увод у информатику}\\[0.5cm]

\hrule
\vspace{0.4cm}
{\LARGE \textbf{Зашто је пиратерија данас постала толико популарна}}
\vspace{0.4cm}
\hrule

\vfill

\noindent
\begin{minipage}[t]{0.45\textwidth}
\textbf{Студент:}\\
Никола Шофранац\\
Бр. индекса: 108/2025
\end{minipage}
\hfill
\begin{minipage}[t]{0.45\textwidth}
\raggedleft
\textbf{Професор:}\\{др Данијела Симић}
\end{minipage}

\vfill

\end{center}

\end{titlepage}
\newpage

\tableofcontents
\newpage

\section{Увод}

Пиратерија у дигиталном окружењу представља један од најраспрострањенијих
феномена савременог доба.
Под појмом \textbf{дигиталне пиратерије} подразумева се
\textit{неовлашћено копирање, дистрибуција и коришћење софтвера, филмова,
музике и других дигиталних садржаја}.

Упркос правним и техничким мерама заштите,
пиратерија је данас \textcolor{red}{лакше доступна него икада раније}.

\section{Основни појмови пиратерије}

\begin{definicija}
Дигитална пиратерија је облик кршења ауторских права који подразумева
неовлашћено умножавање и дељење дигиталних садржаја.
\end{definicija}

\subsection{Историјски развој}

Пиратерија је свој значајнији развој доживела појавом интернета и
широкопојасних мрежа, које су омогућиле брз и јефтин пренос података.

\subsection{Техничка основа}

Количина података која се може пренети зависи од брзине везе, што се може
формално представити као:
\[
D = B \cdot t
\]
где је $D$ количина података, $B$ проток, а $t$ време преноса.

\section{Разлози популарности пиратерије}

Популарност пиратерије може се објаснити комбинацијом економских,
технолошких и друштвених фактора који утичу на понашање корисника
у дигиталном окружењу.

\subsection{Економски фактори}

Економски разлози представљају један од главних узрока ширења пиратерије.
Високе цене оригиналног софтвера, филмова и игара често превазилазе
финансијске могућности просечног корисника.

\begin{itemize}
  \item Висока цена лиценцираног софтвера
  \item Низак животни стандард у појединим земљама
  \item Ограничена доступност легалних сервиса
\end{itemize}

\subsection{Технолошки фактори}

Развој интернета и дигиталних технологија значајно је олакшао
нелегалну дистрибуцију садржаја. Брз пренос података и једноставни алати
за дељење датотека доприносе ширењу пиратерије.

\begin{enumerate}
  \item Брз и стабилан интернет
  \item Торент технологија
  \item Анонимност корисника
\end{enumerate}

\begin{figure}[h]
\centering
\includegraphics[width=0.6\textwidth]{slika.png}
\caption{Piratebay}
\end{figure}

\begin{lema}
Ако је дигитални садржај лако копирати, вероватноћа његове нелегалне
дистрибуције се повећава.
\end{lema}

\section{Последице пиратерије}

\subsection{Економске последице}

\begin{center}
\begin{tabular}{lcc}
\toprule
Индустрија & Губици & Утицај \\
\midrule
Филмска & Високи & Негативан \\
Софтверска & Веома високи & Критичан \\
Музичка & Средњи & Значајан \\
\bottomrule
\end{tabular}
\end{center}

\subsection{Правни аспект}

\begin{teorema}
Свака неовлашћена дистрибуција заштићеног садржаја представља кршење
ауторских права и подлеже законским санкцијама.
\end{teorema}

\section{Закључак}

Пиратерија је постала масовна појава због лаке доступности технологије,
економских разлога и недовољне правне контроле.
Иако краткорочно корисна за појединце, дугорочно има
\textbf{негативан утицај на развој софтвера и креативних индустрија}.

\end{document}


