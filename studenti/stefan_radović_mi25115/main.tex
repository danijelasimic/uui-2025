\documentclass[11pt,a4paper]{article}
\usepackage[a4paper,margin=2cm]{geometry}
\usepackage[utf8]{inputenc}
\usepackage[T2A]{fontenc}
\usepackage[serbian]{babel}
\usepackage{amsmath,amsthm,amssymb} % dodato amssymb
\usepackage{graphicx}
\usepackage{xcolor}

\newtheorem{teorema}{Теорема}
\newtheorem{definicija}{Дефиниција}
\newtheorem{lema}{Лема}

\title{Кратак академски рад}
\author{Стефан Радовић}
\date{\today}

\begin{document}
\maketitle
\tableofcontents

\section{Увод}
Ово је кратак академски текст у \LaTeX-у. Документ је на ћирилици и садржи формуле, табеле, листе, слике и теореме. \textbf{Информатика} и \textit{математика} ме инспиришу, а \textcolor{blue}{креативност} и \textcolor{red}{технологију} желим да спојим.

\section{Математика}
Формула:



\[
E = mc^2
\]



\subsection{Пример теореме}
\begin{teorema}
За свако $n \in \mathbb{N}$ важи $n < n+1$.
\end{teorema}

\subsection{Пример дефиниције}
\begin{definicija}
\emph{Природни бројеви} су $1,2,3,\dots$.
\end{definicija}

\subsection{Пример леме}
\begin{lema}
Ако је $n$ паран, онда је $n^2$ паран.
\end{lema}

\section{Табела и листе}
\begin{tabular}{|c|c|}
\hline
Предмет & Семестар \\ \hline
Увод у програмирање & I \\ 
Анализа 1 & II \\ \hline
\end{tabular}

\begin{enumerate}
\item Први
\item Други
\end{enumerate}
\begin{itemize}
\item Музика
\item Игре
\end{itemize}

\section{Слика}
\includegraphics[width=0.25\textwidth]{slike/slika.png}
Текст поред слике описује моје интересовање за \LaTeX{}.

\section{Закључак}
Приказани су основни елементи: садржај, формуле, табеле, листе, слике и теореме. Циљ је био да се покаже како се може направити академски текст који је истовремено лепо форматиран и функционалан.
\end{document}
