\documentclass[a4paper,12pt]{article}

% =========================
% ПОДРШКА ЗА ЋИРИЛИЦУ
% =========================
\usepackage[T2A]{fontenc}
\usepackage[utf8]{inputenc}
\usepackage[serbian]{babel}

% =========================
% ДИЗАЈН
% =========================
\usepackage{geometry}
\geometry{margin=2.5cm}

\usepackage{xcolor}
\usepackage{graphicx}
\usepackage{wrapfig}
\usepackage{float}
\usepackage{amsmath, amssymb, amsthm}
\usepackage{booktabs}
\usepackage{hyperref}
\usepackage{titlesec}
\usepackage[serbian]{babel}
\usepackage[locale=SR]{datetime2}  % ново
\addto\captionsserbian{%
  \renewcommand{\figurename}{Слика} % за слике
  \renewcommand{\tablename}{Табела} % за табеле
}
\date{\DTMnow}  % користи ДТМ да буде на ћирилици
% Боје
\definecolor{naslov}{RGB}{0,51,102}
\definecolor{naglasak}{RGB}{153,0,0}

% Стил секција


\titleformat{\section}[hang] % hang = лево поравнање са бројем секције
{\color{naslov}\normalfont\Large\bfseries} % формат текста
{\thesection}{1em}{} % број секције и размак
% =========================
% ОКРУЖЕЊА
% =========================
\newtheorem{clan}{Члан}[section]

% =========================
% НАСЛОВ
% =========================
\title{
\textbf{\color{naslov}ЛИЦЕНЦА О РАДУ ПРОДАВНИЦЕ ХАРДВЕР ОПРЕМЕ}
}

\author{
Издавалац: \textbf{Министарство техничке инфраструктуре} \\
Носилац лиценце: \textbf{Hardware Shop}
}

\date{\today}

\begin{document}

% =========================
% ПРВА СТРАНА - НАСЛОВ
% =========================
\begin{titlepage}
\centering
\vspace*{5cm}
{\LARGE \textbf{\color{naslov}ЛИЦЕНЦА О РАДУ ПРОДАВНИЦЕ ХАРДВЕР ОПРЕМЕ} \par}
\vspace{2cm}
{\large Издавалац: \textbf{Министарство техничке инфраструктуре} \par}
\vspace{0.5cm}
{\large Носилац лиценце: \textbf{Hardware Shop} \par}
\vspace{2cm}
{\large Датум издавања: \today \par}
\end{titlepage}

% =========================
% ДРУГА СТРАНА - САДРЖАЈ
% =========================
\renewcommand{\contentsname}{Садржај}
\tableofcontents
\newpage

% =========================
% ТРЕЋА СТРАНА - ТЕКСТ
% =========================
\section{О раду продавнице}

Овај документ представља \textbf{\color{naglasak}званичну лиценцу} за рад продавнице хардвер опреме. Његова сврха је да дефинише техничке, организационе и безбедносне стандарде које продавница мора испунити за легалан рад.  

\noindent
\begin{minipage}{0.38\textwidth}  % лева колона за слику
\includegraphics[width=\linewidth]{hardware.jpg}
\caption{Слика 1: Пример хардвер продавнице}
\end{minipage}%
\hspace{0.5cm}  % размак између слике и текста
\begin{minipage}{0.57\textwidth}  % десна колона за текст, без померања
Продавница се бави продајом:
\begin{itemize}
\item \textbf{процесора} високих перформанси
\item \textit{меморије} великог капацитета
\item \textcolor{naslov}{SSD уређаја} за брз рад система
\item периферних уређаја
\item матичне плоче компатибилне са свим компонентама
\item графичких картица за гејминг и професионални рад
\end{itemize}
\end{minipage}
\section{Технички стандарди и чланови о раду}

\subsection{Члан 1: Процесори}
\begin{clan}
Продавница мора обезбедити да сви продати процесори буду тестирани, исправни и у складу са техничким стандардима.  
\textbf{Испуњено:} Да
\end{clan}

\subsection{Члан 2: Меморија}
\begin{clan}
Меморијски модули морају имати назначене капацитете и радне фреквенције, те бити компатибилни са продајним системима.  
\textbf{Испуњено:} Да
\end{clan}

\subsection{Члан 3: SSD и складишни уређаји}
\begin{clan}
Сви SSD уређаји морају бити тестирани, обезбедити брз приступ подацима и поштовање енергетских стандарда.  
\textbf{Испуњено:} Да
\end{clan}

\subsection{Члан 4: Периферни уређаји}
\begin{clan}
Продаја периферних уређаја као што су тастатуре, мишеви и монитори мора бити уз гаранцију квалитета и компатибилности.  
\textbf{Испуњено:} Да
\end{clan}

\subsection{Стандард савременог рада}
Савремени рад подразумева употребу процесора високих перформанси, меморија великог капацитета и брзих SSD уређаја. Формално, перформансе система могу се изразити формулом:
\[
P = f \times IPC
\]
где је:
\begin{itemize}
\item $P$ — перформанса
\item $f$ — фреквенција процесора
\item $IPC$ — број инструкција по циклусу
\end{itemize}

\section{Организација и евиденција производа}

Продавница мора имати следеће компоненте:
\begin{enumerate}
\item продајни простор
\item складиште
\item информациони систем
\item систем за издавање рачуна
\end{enumerate}

Евиденција производа представљена је у табели:

\begin{table}[H]
\centering
\begin{tabular}{ccc}
\toprule
Производ & Количина & Статус \\
\midrule
CPU & 25 & доступно \\
RAM & 40 & доступно \\
SSD & 15 & ограничено \\
GPU & 10 & доступно \\
\bottomrule
\end{tabular}
\caption{Евиденција производа}
\end{table}



% =========================
% ЧЕТВРТА СТРАНА - ЗАКЉУЧАК
% =========================
\section{Одобрење за рад продавнице}

Овај документ дефинише услове и стандарде за рад продавнице хардвер опреме. Придржавањем ових стандарда, продавница обезбеђује квалитет услуге, сигурност производа и задовољство купаца. Лиценца је важећа све док продавница испуњава све наведене техничке и законске услове.

\vspace{1cm}

\textbf{Датум издавања:} \today

\vspace{0.5cm}

\textbf{Потпис овлашћеног лица:}

\vspace{1cm}

\rule{6cm}{0.4pt}

\textit{Министарство техничке инфраструктуре}

\end{document}
