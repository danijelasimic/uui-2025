\documentclass[a4paper,12pt]{article}

\usepackage{silence}
\WarningFilter{latex}{Command \showhyphens has changed}
\usepackage{hyphenat}
\usepackage[T2A]{fontenc}
\usepackage[utf8]{inputenc}
\usepackage[serbianc]{babel}
\usepackage{amsmath, amsthm}
\usepackage{graphicx}
\usepackage{array}
\usepackage{microtype}  
\usepackage{float}  
\usepackage{wrapfig}
\usepackage[unicode]{hyperref}




\hypersetup{colorlinks,citecolor=green,filecolor=green,linkcolor=blue,urlcolor=blue}

\newtheorem{definicija}{Дефиниција}[section]
\newtheorem{teorema}{Теорема}[section]

\begin{document}

\begin{titlepage}

\newcommand{\HRule}{\rule{\linewidth}{0.4mm}}

\center 
	
\textsc{\LARGE Математички факултет}\\[3cm] 
	
\textsc{\Large Семинарски рад}\\[0.1cm]
	
\textsc{\large из увода у информатику}\\[0.5cm] 
	
\HRule\\[0.4cm]
	
{\LARGE\bfseries Bubble sort}\\[0.2cm] 
	
\HRule\\[2cm]

\vspace{17\baselineskip}
		

	
	\begin{minipage}{0.4\textwidth}
		\begin{flushleft}
			\large
			\textit{Студент}\\
			Стефан Спасић 33/2025
		\end{flushleft}
	\end{minipage}
\hspace*{1cm}
	\begin{minipage}{0.4\textwidth}
		\begin{flushright}
			\large
			\textit{Професор}\\
			др Данијела Симић
		\end{flushright}
	\end{minipage}
	
	
\vfill\vfill\vfill\vfill 
	
{\large Београд, \today} 
	
\vfill 
	
\end{titlepage}

\tableofcontents

\newpage


\section{Увод}

\textbf{Bubble Sort} (енг.~\textit{bubble sort}) је једноставан алгоритам
сортирања који више пута пролази кроз низ и упоређује суседне елементе.

\textcolor{blue}{Основна идеја} алгоритма је да већи елементи „испливају“
на крај низа.

\section{Опис алгоритма}

\begin{definicija}
Bubble Sort је алгоритам који понавља поређење суседних елемената
и замењује их ако нису у правилном поретку.
\end{definicija}

Алгоритам се састоји из следећих корака:

\subsection{Кораци алгоритма}

\begin{enumerate}
    \item Проћи кроз цео низ.
    \item Упоредити суседне елементе.
    \item Ако је леви већи од десног, извршити замену.
    \item Поновити поступак док низ не буде сортиран.
\end{enumerate}

\subsection{Особине алгоритма}

\begin{itemize}
    \item Једноставан за имплементацију
    \item Стабилан алгоритам
    \item Неефикасан за велике скупове података
\end{itemize}

\section{Математичка анализа}

Број поређења у најгорем случају:

\[
(n - 1) + (n - 2) + \dots + 1 = \frac{n(n-1)}{2}
\]

Временска сложеност је:

\[
O(n^2)
\]

\begin{teorema}
У најбољем случају, када је низ већ сортиран,
временска сложеност алгоритма је $O(n)$.
\end{teorema}

\section{Пример и табела}

Нека је дат низ $[5, 3, 8, 4]$.

\begin{table}[h!]
\centering
\caption{Пролази алгоритма}
\begin{tabular}{|c|c|}
\hline
Пролаз & Стање низа \\ \hline
1 & 3 5 4 8 \\ \hline
2 & 3 4 5 8 \\ \hline
\end{tabular}
\end{table}

\section{Илустрација}

\begin{wrapfigure}{l}{0.45\textwidth}
\centering
\includegraphics[width=0.42\textwidth]{images/slika.jpg}
\caption{Слика алгоритма}
\label{fig:bubble}
\end{wrapfigure}

Слика приказује симболичан начин функционисања алгоритма Bubble Sort.
На сваком пролазу алгоритам пореди суседне елементе и врши њихову
замену уколико нису у правилном поретку. 

Већи елементи постепено „испливавају“ ка крају низа,
док се мањи померају ка почетку. Овај процес се понавља
све док се низ у потпуности не сортира.

Иако је алгоритам једноставан за разумевање и имплементацију,
његова временска сложеност $O(n^2)$ чини га неефикасним
за велике скупове података.


\section{Закључак}

Bubble Sort је \textbf{једноставан} и \textit{образовно користан} алгоритам.
Иако има \textcolor{red}{квадратну временску сложеност},
често се користи у настави ради разумевања основа сортирања.

\end{document}
