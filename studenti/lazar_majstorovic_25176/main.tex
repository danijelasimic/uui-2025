\documentclass[a4paper,12pt]{article}

\usepackage[utf8]{inputenc}
\usepackage[T2A]{fontenc}
\usepackage[serbian]{babel}
\usepackage{graphicx}
\usepackage{amsmath}
\usepackage{amsthm}
\usepackage{xcolor}
\usepackage{hyperref}
\usepackage{geometry}
\usepackage{microtype}

\geometry{
 a4paper,
 total={170mm,257mm},
 left=20mm,
 top=20mm,
}

% Stabilizacija preloma teksta
\emergencystretch=3em
\hyphenpenalty=500
\tolerance=1000

\newtheorem{theorem}{Теорема}[section]
\newtheorem{definition}{Дефиниција}[section]
\newtheorem{lemma}{Лема}[section]

\title{\textbf{Истраживање свемира: Пут ка звездама}}
\author{Студент Лазар Мајсторовић 176/2025}
\date{\today}

\begin{document}

\maketitle
\tableofcontents
\newpage

\section{Увод}

Истраживање свемира је откривање и истраживање небеских структура у свемиру помоћу \textbf{развијене свемирске технологије}. Док се проучавање свемира обавља углавном помоћу телескопа са Земље, физичко истраживање се врши помоћу роботских сонди и летова са људском посадом.

\section{Основе свемирске физике}

\subsection{Ракетна једначина}

Основна једначина која описује кретање ракете је \textit{Циолковскијева ракетна једначина}. Она показује везу између промене брзине летелице ($\Delta v$), брзине издувних гасова ($v_e$), као и односа почетне масе ($m_0$) и крајње масе ($m_f$):

\begin{equation}
\Delta v = v_e \ln \frac{m_0}{m_f}
\end{equation}

Такође, не смемо заборавити Ајнштајнову еквиваленцију масе и енергије:

\begin{equation}
E = mc^2
\end{equation}

\section{Математички модел}

\begin{definition}
\textbf{Орбита} је гравитационо закривљена путања објекта, као што је путања планете око звезде или природног сателита око планете.
\end{definition}

\begin{theorem}
Свака планета се креће око Сунца по елипси, при чему се Сунце налази у једној од жижа те елипсе (Први Кеплеров закон).
\end{theorem}

\begin{lemma}
Површина коју описује радијус-вектор планете у једнаким временским интервалима остаје константна.
\end{lemma}

\section{Значајне мисије}

\begin{figure}[h!]
    \centering
    \includegraphics[width=0.4\textwidth]{raketa.jpg}
    \caption{Симбол свемирске мисије}
\end{figure}

Током историје било је много значајних мисија. \textcolor{blue}{Аполо програм} представља један од најпознатијих подухвата у историји човечанства.

\subsection{Табеларни преглед}

У наставку је приказана табела неких кључних догађаја:

\begin{center}
\begin{tabular}{|c|c|c|}
\hline
\textbf{Година} & \textbf{Мисија} & \textbf{Опис} \\ 
\hline
1957 & Спутњик 1 & Први вештачки сателит \\  
\hline
1961 & Восток 1 & Први човек у свемиру \\
\hline
1969 & Аполо 11 & Прво слетање на Месец \\
\hline
\end{tabular}
\end{center}

\section{Технологије}

Свемирске технологије обухватају:

\begin{itemize}
    \item Ракете носаче
    \item Сателите
    \item Свемирске станице
\end{itemize}

Типови погона укључују:

\begin{enumerate}
    \item Хемијски погон
    \item Јонски погон
    \item Нуклеарни термални погон
\end{enumerate}

\section{Закључак}

\textbf{Истраживање свемира} омогућава нам да боље разумемо наше место у универзуму и историју соларног система. Будућност доноси нове изазове, укључујући мисије на Марс и истраживање удаљенијих делова свемира.

\end{document}