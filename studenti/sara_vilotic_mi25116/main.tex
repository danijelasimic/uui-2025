\documentclass[12pt,a4paper]{article}

% ================ Пакети =================
\usepackage[T2A]{fontenc}
\usepackage[serbian]{babel}
\usepackage[utf8]{inputenc}
\usepackage{amsmath, amsthm, amssymb}
\usepackage{graphicx}
\usepackage{xcolor}
\usepackage{hyperref}
\usepackage{enumitem}
\usepackage{geometry}
\usepackage{float}
\usepackage{caption}
\geometry{a4paper, margin=2.5cm}

% ================== ТЕОРЕМЕ ==================
\newtheorem{definicija}{Дефиниција}[section]
\newtheorem{teorema}{Теорема}[section]
\newtheorem{lema}{Лема}[section]

\addto\captionsserbian{%
  \renewcommand{\figurename}{Слика}%
  \renewcommand{\tablename}{Табела}%
  \renewcommand{\contentsname}{Садржај}%
}

\title{\textbf{Фибоначијев низ и његове примене}}
\author{{Сара Вилотић} \\
Математички факултет, Универзитет у Београду \\
Број индекса: 116/2025
}
\date{фебруар, 2026.}

\begin{document}

\maketitle
\tableofcontents
\newpage

\section{Увод}
Фибоначијев низ представља један од најпознатијих нумеричких низова у математици. Његова примена се може уочити у:
\begin{itemize}
    \item природи
    \item уметности
    \item архитектури
    \item информатици
\end{itemize}

У овом раду биће дати основни појмови, формуле и неки примери.

% ================== ОСНОВНИ ПОЈМОВИ ==================
\section{Основни појмови}

\begin{definicija}
Фибоначијев низ је дефинисан рекурзивном формулом:
\[
F_n = F_{n-1} + F_{n-2}, \quad n \geq 2
\]
при чему су почетни чланови:
\[
F_0 = 0, \quad F_1 = 1, \quad F_2 = 1.
\]
\end{definicija}

\begin{table}[h!]
\centering
\begin{tabular}{|c|c|}
\hline
$n$ & $F_n$ \\
\hline
0 & 0 \\
1 & 1 \\
2 & 1 \\
3 & 2 \\
4 & 3 \\
5 & 5 \\
6 & 8 \\
7 & 13 \\
8 & 21 \\
9 & 34 \\
\hline
\end{tabular}
\caption{Првих 10 Фибоначијевих бројева}
\label{tab:fibonacci10}
\end{table}

\subsection{Математичке особине}
За велике вредности, $n$ тежи броју:
\[
\varphi = \frac{1 + \sqrt{5}}{2} = 1.6180339887...
\]


\begin{lema}
Фибоначијев низ расте експоненцијално.
\end{lema}

\begin{teorema}
Однос суседних Фибоначијевих бројева конвергира ка златном пресеку:
\[
\lim_{n \to \infty} \frac{F_{n+1}}{F_n} = \phi.
\]
\end{teorema}

\subsection{Фибоначијева спирала}

\begin{figure}[H]
\begin{minipage}{0.45\textwidth}
\centering
\includegraphics[width=\textwidth]{fibonacijev.jpg}
\captionsetup{justification=raggedright, singlelinecheck=false}
\caption{Фибоначијева спирала}
\label{fig:fibonacci_spiral}
\end{minipage}%
\hfill
\begin{minipage}{0.5\textwidth}
\textcolor{red}{\textbf{Фибоначијева спирала}} представља визуелну аппроксимацију златне спирале. 
Она се често користи као симбол хармоније и равнотеже у природи и уметности. 
Овај облик је један од најпознатијих математичких симбола.
\end{minipage}
\end{figure}

\section{Историјат Фибоначијевог низа}

\begin{figure}[H]
\begin{minipage}{0.5\textwidth}
Фибоначијев низ је увео италијански математичар \textcolor{blue}{\textbf{Леонардо из Пизе}}, познатији као
\textcolor{blue}{\textbf{Фибоначи}}, у 13. веку.  
Он је овај низ представио у својој књизи \textit{Liber Abaci} (1202), кроз проблем
размножавања зечева.

Проблем је гласио: ако један пар зечева сваког месеца добија нови пар, колико ће
парова бити након одређеног броја месеци?  
Решење овог проблема довело је до низа у коме је сваки број збир претходна два,
данас познатог као Фибоначијев низ. \\

Фибоначи се у свом раду бавио различитим областима математике и науке, међу којима су:

\begin{enumerate}
    \item аритметика и рачунање са бројевима
    \item алгебра и решавање једначина
    \item геометрија
    \item теорија бројева
    \item примена математике у трговини и финансијама
    \item ширење арапских бројева у Европи
\end{enumerate}

\end{minipage}%
\hfill
\begin{minipage}{0.45\textwidth}
\centering
\includegraphics[width=\textwidth]{Fibonaci.jpg}
\captionsetup{justification=raggedright, singlelinecheck=false}
\caption{Леонардо Фибоначи}
\label{fig:fibonacci}
\end{minipage}
\end{figure}

\begin{figure}[H]
\centering
\includegraphics[width=0.6\textwidth]{zecevi.jpg}
\captionsetup{justification=centering}
\caption{Илустрација проблема размножавања зечева који је довео до Фибоначијевог низа}
\label{fig:zecevi}
\end{figure}


\section{Закључак}
Фибоначијев низ представља један од најпознатијих и најинтересантнијих низова у математици.  
Његова примена у природним и уметничким структурама показује лепоту и универзалност математике.  

\end{document}
