\documentclass{article}
\usepackage[utf8]{inputenc}
\usepackage[T2A]{fontenc}
\usepackage[serbian]{babel}
\usepackage{amsmath, amsthm, amssymb}
\usepackage{graphicx}
\usepackage{color}
\usepackage{geometry}
\usepackage{datetime}
\geometry{margin=2cm}

\title{\textbf{Емулaција PlayStation конзола}}
\author{Мина Немет}
\newdateformat{mydate}{\twodigit{\THEDAY}.\twodigit{\THEMONTH}.\THEYEAR}

\date{\mydate\today}

\addto\captionsserbian{%
  \renewcommand{\contentsname}{Садржај}%
}


% Definisanje teoreme, definicije i leme
\newtheorem{teorema}{Теорема}[section]
\newtheorem{definicija}{Дефиниција}[section]
\newtheorem{lema}{Лема}[section]

\begin{document}

\maketitle

\tableofcontents
\newpage

\section{Увод}
Емулaција \textbf{PlayStation} конзола постаје све релевантнији хоби с обзиром на то да је људима ефикасније и економичније да ретро игрице покрећу помоћу емулатора на својим локалним машинама. \textit{Емулатори} управо омогућавају такво покретање старих игара на савременим рачунарима.

\section{Технички аспект емулaције}
\subsection{Хардверска емулација}
Емулaтори морају да симулирају \textcolor{blue}{CPU}, \textcolor{red}{GPU} и меморију конзоле.

\subsection{Софтверска емулација}
\begin{itemize}
    \item Интерпретатори инструкција
    \item Управљачи улазних уређаја
\end{itemize}

\subsection{Табела карактеристика конзола}
\begin{center}
\begin{tabular}{|c|c|c|}
\hline
Конзола & CPU фреквенција & GPU меморија \\
\hline
PS1 & 33.8688 MHz & 2 MB \\
PS2 & 294 MHz & 4 MB \\
PS3 & 3.2 GHz & 256 MB \\
\hline
\end{tabular}
\end{center}

\section{Емулaција у пракси}
\subsection{Примери емулатора}
\begin{enumerate}
    \item \textbf{PCSX} --- за PS1
    \item \textbf{PCSX2} --- за PS2
    \item \textbf{RPCS3} --- за PS3
\end{enumerate}

\subsection{Илустрација}
\begin{figure}[h!]
\flushleft
\includegraphics[width=0.4\textwidth]{slike/ps1.jpg}
\caption{Симболична слика PlayStation конзоле}
\end{figure}

\section{Математички модел}
Емулaција често захтева пресликавање физичких својстава конзоле у дигитални облик. На пример, формула за рачунање циклуса процесора емулатора на основу процесора изворне конзоле могла би бити:
\[
\text{emulator\_cycles} = \text{console\_cycles} \times \frac{\text{emulator\_clock\_speed}}{\text{console\_clock\_speed}}
\]

\section{Теоремe и дефиниције}
\begin{definicija}
Емулaтор је софтвер који омогућава покретање програма намењених другом хардверском окружењу.
\end{definicija}

\begin{lema}
Сваки \textit{инструкцијски скуп} оригиналне конзоле може бити симулиран на рачунару са довољно меморије и процесорске снаге.
\end{lema}

\begin{teorema}
Ако емулaтор правилно симулира CPU, GPU и меморију, све оригиналне игре ће се покретати без грешака.
\end{teorema}

\section{Закључак}
Емулaција PlayStation конзола је комплексна комбинација \textbf{хардверских} и \textit{софтверских} решења. Иако постоје изазови, као што су \textcolor{blue}{перформансе} и компатибилност, емулaтори омогућавају очување дигиталне културне баштине.

\end{document}
