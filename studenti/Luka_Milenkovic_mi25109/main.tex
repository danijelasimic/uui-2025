\documentclass[a4paper,12pt]{article}

\usepackage[T2A]{fontenc}
\usepackage[utf8]{inputenc}
\usepackage[serbianc]{babel}

\usepackage{amsmath,amsthm}
\usepackage{graphicx}
\usepackage{wrapfig}
\usepackage{xcolor}
\usepackage{geometry}
\usepackage{enumitem}
\usepackage{caption}
\usepackage[colorlinks=true, linkcolor=blue]{hyperref}

\geometry{margin=2.5cm}

\newtheorem{definicija}{Дефиниција}
\newtheorem{teorema}{Теорема}
\newtheorem{lema}{Лема}

\begin{document}

\begin{titlepage}
\centering

\vspace*{4cm}  

{\LARGE Математички факултет}\\[2cm]
{\Large Семинарски рад}\\[0.3cm]
{\large Увод у информатику}\\[1cm]

\rule{\linewidth}{0.4mm}\\[0.4cm]
{\LARGE\bfseries Питагорина теорема}\\[0.4cm]
\rule{\linewidth}{0.4mm}

\vspace*{\fill}  

\large
Лука Миленковић \\[0.5cm]
Београд, \today

\end{titlepage}

\tableofcontents
\newpage

\section{Увод}

\begin{wrapfigure}{l}{0.4\textwidth}
    \centering
    \includegraphics[width=0.35\textwidth]{slike/pitagora.jpg}
    \caption{Геометријски доказ питагорине теореме}
\end{wrapfigure}

\textbf{Питагорина теорема} представља једну од најпознатијих теорема у математици. 
Она описује однос страница у \textit{правоуглом троуглу}. 
Теорема је добила име по старогрчком математичару Питагори.

Основна формула гласи:
\[
a^2 + b^2 = c^2
\]

где су $a$ и $b$ катете, а $c$ хипотенуза правоуглог троугла.

\section{Математичке основе}

\subsection{Дефиниције}

\begin{definicija}
Правоугли троугао је троугао који има један угао од $90^\circ$.
\end{definicija}

\begin{definicija}
Хипотенуза је страница правоуглог троугла која лежи наспрам правог угла.
\end{definicija}

\subsection{Лема и теорема}

\begin{lema}
Површина квадрата странице $a$ једнака је $a^2$.
\end{lema}

\begin{proof}
Површина квадрата израчунава се као производ странице са самом собом.
\end{proof}

\begin{teorema}[Питагорина теорема]
У правоуглом троуглу збир квадрата над катетама једнак је квадрату над хипотенузом:
\[
a^2 + b^2 = c^2.
\]
\end{teorema}

\begin{proof}
Конструишемо квадрат над сваком страницом правоуглог троугла. 
Збир површина квадрата над катетама једнак је површини квадрата над хипотенузом. 
Доказ се може извести геометријски или алгебарски.
\end{proof}

\subsection{Кораци рачунања}
Листа корака рачунања:
\begin{enumerate}
    \item Квадрирати катете
    \item Сабрати добијене вредности
    \item Израчунати квадратни корен
\end{enumerate}

На пример, ако је $a = 3$ и $b = 4$, онда је:
\[
c = \sqrt{3^2 + 4^2} = \sqrt{9 + 16} = \sqrt{25} = 5.
\]

\section{Питагорине тројке}

Питагорина тројка је уређена тројка природних бројева a, b, и c за које важи Питагорина теорема. Бројеви за које важи дата релација се обично записују у облику (a, b, c). Један од најстаријих познатих примера је тројка (3, 4, 5), која се може наћи у заоставштини древних Вавилонаца и Египћана.

Уколико је (a, b, c) Питагорина тројка, онда је то и свака тројка облика (ka, kb, kc) где је k произвољан природан број. Основна Питагорина тројка је тројка у којој су бројеви a, b и c узајамно прости. 

\subsection{Пример Питагориних тројки}

\begin{center}
\begin{tabular}{|c|c|c|}
\hline
$a$ & $b$ & $c$ \\ 
\hline
3 & 4 & 5 \\ 
5 & 12 & 13 \\ 
8 & 15 & 17 \\ 
\hline
\end{tabular}
\end{center}

\section{Примена}

\begin{itemize}
    \item Геометрија
    \item Архитектура
    \item Физика
\end{itemize}

\section{Закључак}

Питагорина теорема представља \textbf{основни резултат еуклидске геометрије}. 
Њена примена је широка и обухвата бројне области науке и технике. 
\textcolor{blue}{Њена једноставност и елеганција} чине је једним од 
најлепших резултата математике.

\end{document}
