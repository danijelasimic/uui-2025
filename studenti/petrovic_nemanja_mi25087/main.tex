\documentclass[12pt,a4paper]{article}

% Paketi za podršku jeziku i pismu
\usepackage[T2A]{fontenc}
\usepackage[utf8]{inputenc}
\usepackage[serbianc]{babel}

% Paketi za matematiku, grafiku i boje
\usepackage{amsmath, amsthm, amssymb}
\usepackage{graphicx}
\usepackage{wrapfig}
\usepackage{xcolor}
\usepackage{booktabs}
\usepackage{hyperref}

% Podešavanje hiperlinkova da budu crni (akademski izgled)
\hypersetup{
    colorlinks=true,
    linkcolor=black,
    filecolor=black,      
    urlcolor=blue,
}

% Definisanje okruženja na srpskom (ćirilica)
\theoremstyle{definition}
\newtheorem{definicija}{Дефиниција}[section]
\theoremstyle{plain}
\newtheorem{teorema}{Теорема}[section]
\newtheorem{lema}{Лема}[section]

% Podaci o dokumentu
\title{\textbf{Теорија мртвог интернета}\\ \large Анализа аутоматизованог мрежног саобраћаја}
\author{Немања Петровић, 87/2025}
\date{\today}

\begin{document}

\maketitle

\renewcommand{\contentsname}{Садржај}
\tableofcontents
\vspace{1cm}

\section{Увод у теорију}

\begin{wrapfigure}{l}{0.4\textwidth}
  \vspace{-10pt}
  \begin{center}
    % NAPOMENA: Slika mora postojati u folderu 'slike'
    \includegraphics[width=0.38\textwidth]{slike/tema-bg.jpg}
  \end{center}
  \vspace{-15pt}
  \caption{Мрежа конекција}
  \vspace{-10pt}
\end{wrapfigure}

\textbf{Теорија мртвог интернета} (енгл. \textit{Dead Internet Theory}) је концепт који сугерише да је \textcolor{blue}{већина садржаја на интернету генерисана од стране алгоритама} и ботова, а не од стране стварних људи. Ова теорија је постала посебно актуелна развојем напредних језичких модела.

Према овој теорији, интернет какав познајемо пролази кроз фазу симулације, где се ствара \textit{илузија људске активности} кроз континуирану интеракцију различитих алгоритама.

\section{Анализа и математички модел}

Да бисмо боље разумели ову појаву, неопходно је формализовати концепте рачунарског саобраћаја.

\begin{definicija}
\textbf{Ботовски саобраћај} представља скуп свих мрежних захтева $B$ у укупном саобраћају $U$, таквих да су иницирани аутоматизованим скриптама без директне људске интервенције.
\end{definicija}

Можемо поставити математички модел раста количине вештачког садржаја у времену $t$. Нека је $C_{ai}(t)$ количина аутоматизованог садржаја. Њен раст се често моделује експоненцијалном функцијом:
\begin{equation}
    C_{ai}(t) = C_0 \cdot e^{k \cdot t}
\end{equation}
где је $C_0$ почетна количина садржаја, а $k$ константа која представља стопу технолошког напретка. Уколико посматрамо удео ботовског саобраћаја у укупном саобраћају $C_{total}$, добијамо граничну вредност:

\begin{equation}
    \lim_{t \to \infty} \frac{C_{ai}(t)}{C_{total}(t)} = 1
\end{equation}

\begin{lema}
Ако стопа генерисања аутоматизованог садржаја $k_{ai}$ строго доминира над стопом људског генерисања $k_{h}$ ($k_{ai} > k_{h}$), удео људског садржаја тежи нули како време тежи бесконачности.
\end{lema}

\begin{teorema}[О асимптотској доминацији]
У затвореном систему размене информација, уколико је капацитет производње података од стране вештачке интелигенције експоненцијалан, свака линеарна или полиномијална продукција од стране људи биће асимптотски занемарљива.
\end{teorema}

\subsection{Статистички подаци}

У табели испод су приказане процене присуства ботова на популарним платформама, базиране на доступним истраживањима.

\begin{table}[h]
\centering
\begin{tabular}{lcc}
\toprule
\textbf{Платформа} & \textbf{Проценат ботова} & \textbf{АИ садржај} \\
\midrule
Twitter/X & $\approx 15\%$ & $\approx 25\%$ \\
Facebook & $\approx 20\%$ & $\approx 30\%$ \\
Instagram & $\approx 10\%$ & $\approx 35\%$ \\
YouTube & \textcolor{red}{$\approx 25\%$} & \textcolor{red}{$\approx 40\%$} \\
\bottomrule
\end{tabular}
\caption{Процена удела аутоматизованог саобраћаја по платформама}
\end{table}

\section{Аргументи и импликације}

Расправа о тачности ове теорије води се кроз различите аргументе.

\textbf{Аргументи који подржавају теорију (неуређена листа):}
\begin{itemize}
    \item Експлозија генерисаног садржаја након појаве великих језичких модела.
    \item Преваленција \textit{spam} ботова и аутоматизованих коментара.
    \item Пад квалитета и релевантности резултата на интернет претраживачима.
\end{itemize}

\textbf{Кључне импликације за друштво (уређена листа):}
\begin{enumerate}
    \item \textbf{Дезинформације:} Брзо ширење непроверених информација.
    \item \textbf{Приватност:} Масовно прикупљање података од стране мрежних паука.
    \item \textbf{Аутентичност:} Губитак поверења у дигиталне идентитете.
\end{enumerate}

\section{Закључак}

Иако се Теорија мртвог интернета често посматра као радикална, она служи као важно упозорење. Разумевање алгоритама који обликују нашу дигиталну стварност кључно је за очување људске аутентичности у онлајн простору. Математички модели јасно указују на то да без филтрирања и регулативе, \textcolor{blue}{\textbf{вештачки генерисан садржај неминовно постаје доминантан}}.

\end{document}