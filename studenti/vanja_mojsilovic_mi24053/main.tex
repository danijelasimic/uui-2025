
\documentclass[12pt]{article}

\usepackage[T2A]{fontenc}
\usepackage[utf8]{inputenc}
\usepackage[serbian]{babel}
\usepackage{amsmath, amsthm}
\usepackage{geometry}
\usepackage{xcolor}
\usepackage{graphicx}  % za slike
\usepackage{hyperref}  % za linkove
\usepackage{float}     % precizno pozicioniranje slika
\usepackage{titlesec}

\geometry{margin=2.5cm}

\hypersetup{
    colorlinks=true,
    linkcolor=blue,
    urlcolor=cyan,
    citecolor=blue
}

\title{\textbf{Ојлерови графови и Флеријев алгоритам}}
\author{Вања Мојсиловић}
\date{\today}

\theoremstyle{definition}
\newtheorem{definition}{Дефиниција}
\newtheorem{theorem}{Теорема}
\newtheorem{lemma}{Лема}

% Glavni naslovi: plavi i veći
\titleformat{\section}[block]
  {\normalfont\bfseries\LARGE\color{blue}}{\thesection}{1em}{}

% Podnaslovi: crni i normalne veličine
\titleformat{\subsection}[runin]{\normalfont\bfseries\color{black}}{\thesubsection}{1em}{}

\begin{document}
\maketitle
\tableofcontents
\newpage

\section{Ојлерови мултиграфови}

\begin{definition}
\textbf{Мултиграф} је граф који може да садржи више од једне гране између истих чворова.
\end{definition}

За мултиграф кажемо да је \textbf{Ојлеров} ако садржи затворену стазу која сваком граном пролази једном. Такву стазу називамо \emph{Ојлерова стаза}. Мултиграф је \emph{полуојлеров} ако садржи стазу која сваком граном пролази једном. Јасно је да сваки Ојлеров мултиграф истовремено представља и полуојлеров.

Ојлеров пут је пут у графу који обилази сваку грану тачно једном, а Ојлеров циклус почиње и завршава се у истом чвору.

\begin{theorem}[Ојлерова теорема]
Нетривијалан повезан мултиграф је \textbf{Ојлеров} ако и само ако су сви чворови парног степена.
\end{theorem}

\begin{theorem}
Нетривијалан повезан мултиграф је полуојлеров ако и само ако садржи 0 или 2 чвора непарног степена.
\end{theorem}

\begin{lemma}
Сваки Ојлеров граф је такође полуојлеров.
\end{lemma}

\subsection{Седам Кенигсбершких мостова}

Ојлер је 1741. године објавио научни рад о 7 кенингсбершких мостова, који се сматра и првим радом из теорије графова.  
Проблем може бити формулисан математички:

С обзиром на граф на слици, да ли је могуће да се изгради пут (или циклус, то је пут који почиње и завршава се у истом чвору), који посети сваку грану тачно једном?  

Ојлер је проблем решио тако што је конструисао припадни мултиграф чији чворови одговарају обалама реке и речним острвима, а гране мостовима. Два чвора мултиграфа спојена су са онолико грана колико мостова спаја одговарајуће делове града. Тај мултиграф илустрован је на сликама испод:

\begin{figure}[H]
\centering
\fbox{\includegraphics[width=0.45\textwidth]{slike/sedam_mostova.jpeg}}
\caption{\emph{Седам Кенигсбершких мостова}}
\end{figure}

\begin{figure}[H]
\centering
\fbox{\includegraphics[width=0.45\textwidth]{slike/most.png}}
\caption{\emph{Припадни мултиграф седам мостова}}
\end{figure}

\subsection{Ојлерова формула у графовима}

У контексту Ојлерових графова и планарног графа можемо користити \textbf{Ојлерову формулу}:

\[
V - E + F = 2
\]

gde su:
\begin{itemize}
    \item $V$ -- број чворова (\textcolor{blue}{vertices}),
    \item $E$ -- број grana (\textcolor{green}{edges}),
    \item $F$ -- број лица (faces) у планарном графу.
\end{itemize}

Ова формула је корисна за анализу структуре графова и одређивање да ли одређени граф може имати Ојлерову стазу или циклус.

\section{Флеријев алгоритам}

Флеријев алгоритам проналази Ојлеров пут или циклус у мултиграфу. Основни кораци су:

\begin{itemize}
    \item Изабрати произвољан чвор $v_0$, ако тражимо \textbf{Ојлеров пут}, $v_0$ мора бити један од два чвора непарног степена.
    \item Додајемо гране у стазу, избегавајући мостове осим ако нема друге опције.
    \item Понављамо док све гране не буду коришћене.
\end{itemize}

\begin{center}
\begin{tabular}{|c|p{10cm}|}
\hline
K & Опис корака \\
\hline
К1 & Изабрати произвољан чвор $v_0$, $w=v_0$ (ако тражимо \textbf{Ојлеров пут}, $v_0$ мора бити један од два чвора непарног степена) \\
К2 & Нека је одабрана стаза $w=v_0, e_1, v_1, e_2, \dots, e_i, v_i$. Наредну грану $e_{i+1}$ одабрати из скупа $E \setminus \{e_1, e_2, \dots, e_i\}$ тако да је $e_{i+1}$ инцидентна са $v_i$ и при томе није мост графа $G_i = G - \{e_1, e_2, \dots, e_i\}$, осим ако нема друге могућности. \\
К3 & КРАЈ када К2 не може да се извршава. \\
\hline
\end{tabular}
\end{center}

\textbf{Напомена:} Кључно правило алгоритма је "\textbf{не прелази мост осим ако мораш}", како би се осигурало да не останеш заробљен у једном делу графа док у другом још увек има неискоришћених грана.

\section{Закључак}

Ојлерови и полуојлерови мултиграфови, као и Флеријев алгоритам, имају значајну примену у практичним проблемима теорије графова и алгоритамском планирању. Користе се за \textbf{оптимизацију рута}, \textbf{планирање мрежа} и различите алгоритамске задатке.

\section*{Литература}
\begin{itemize}
\item З. Станић, \textit{Дискретне структуре 2}, Математички факултет, Универзитет у Београду.\\
\href{http://poincare.matf.bg.ac.rs/~zstanic//ds2/DS2-2021.pdf}{\textcolor{cyan}{http://poincare.matf.bg.ac.rs/~zstanic//ds2/DS2-2021.pdf}}
\item \href{http://old.matf.bg.ac.rs/p/jelena/cas/4399/materijali-za-vezbe/}{\textcolor{cyan}{http://old.matf.bg.ac.rs/p/jelena/cas/4399/materijali-za-vezbe/}}
\item \href{https://en.wikipedia.org/wiki/Eulerian_path}{\textcolor{cyan}{https://en.wikipedia.org/wiki/Eulerian_path}}
\end{itemize}

\end{document}