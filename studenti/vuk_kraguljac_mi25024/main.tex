\documentclass[a4paper]{article}
\usepackage{graphicx} % Required for inserting images
\usepackage{fontspec}
\setmainfont{DejaVu Serif}
\usepackage[colorlinks=true, linkcolor=black, urlcolor=blue, citecolor=black]{hyperref}
\usepackage{polyglossia}
\setdefaultlanguage{serbian}
\usepackage{amsmath, amsthm}
\usepackage{amssymb}


\newtheorem{teorema}{Теорема}
\newtheorem{lema}{Лема}
\newtheorem{definicija}{Дефиниција}

\title{Кардиналност скупова}
\author{Вук Крагуљац}

\begin{document}

\begin{titlepage}

\newcommand{\HRule}{\rule{\linewidth}{0.4mm}}

\center 
	
\textsc{\LARGE Математички факултет}\\[3cm] 
	
\HRule\\[0.4cm]
	
{\LARGE\bfseries Кардиналност скупова}\\[0.2cm] 
	
\HRule\\[2cm]

\vspace{17\baselineskip}
		
\begin{minipage}{0.4\textwidth}
    \begin{flushleft}
        \large
        \textit{Студент}\\
        Вук Крагуљац 24/2025
    \end{flushleft}
\end{minipage}
\begin{minipage}{0.4\textwidth}
    \begin{flushright}
        \large
        \textit{Професор}\\
        др Данијела Симић
    \end{flushright}
\end{minipage}
	
\vfill
	
{\large Београд, \today} 
	
\vfill 
	
\end{titlepage}

\tableofcontents
\newpage

\begin{abstract}
Кардиналност скупа представља \textbf{меру величине} скупа, односно број његових елемената.  
Појам кардиналности је од фундаменталног значаја у савременој математици, нарочито у теорији скупова.

У овом раду разматрамо основне идеје и појмове кардиналности, као што су:
\begin{itemize}
    \item Коначни скупови
    \item Пребројиви скупови
    \item Непребројиви скупови
\end{itemize}
као и разлике између различитих врста бесконачности.

\end{abstract}

\section{Пребројиви скупови}
\begin{definicija}
Непразан скуп $A$ је \textit{коначан} ако постоји природан број $n$ такаб да постоји бијекција $f: A \to \mathbb{N}_{n}$ . Тада $A$ има n елемената (у ознаци $|A| = n$). Ако скуп није коначан, онда је бесконачан.
\end{definicija}
\begin{definicija}
Скуп $A$ је пребројив ако постоји бијекција \\ \mbox{$f: A \to \mathbb{N}$} . Ако је скуп коначан или пребројив, онда је \textit{највише пребројив}, иначе је непребројив.
\end{definicija}

\section{Кардиналност скупова}
\begin{definicija}
Скуп $A$ је мање или једнаке кардиналности од скупа $B$ (у ознаци $|A| \leq |B|$) ако постоји инјективна функција из $A$ у $B$. Скупови $A$ и $B$ су исте кардиналности (у ознаци $|A| = |B|$) ако постоји бијекција између њих.
\end{definicija}
\begin{teorema}
\textcolor{red}{Кантор-Бернштајнова теорема}: Ако постоји инјекција из $A$ у $B$ и инјекција из $B$ у $A$, тада постоји бијекција између та два скупа.
\end{teorema}
\begin{lema}
Нека је $A \subseteq B \subseteq C$ и $|A| = |C|$. Тада је $|B| = |A|$.
\end{lema}

\newpage
\subsection{Примери кардиналности}

\begin{table}[h]
\centering
\caption{Примери скупова и њихових кардиналности}
\begin{tabular}{|c|c|}
\hline
Скуп & Кардиналност \\
\hline
$\{1,2,3\}$ & $3$ \\
$\mathbb{N}$ & $\aleph_0$ \\
$\mathbb{R}$ & $2^{\aleph_0}$ \\
\hline
\end{tabular}
\end{table}

Скуп природних бројева и скуп целих бројева су исте кардиналности.
\begin{figure}[h]
\begin{minipage}{0.35\linewidth}
    \includegraphics[width=\linewidth]{slike/z_u_n.jpg}
\end{minipage}\hfill
\begin{minipage}{0.6\linewidth}
    На слици је приказана бијекција која слика скуп $\mathbb{N}$ у $\mathbb{Z}$
\end{minipage}
\end{figure}

\section{Закључак}
У овом кратком раду анализирали смо кардиналност скупова, основне дефиниције, примере и важне теореме.  
Посебно је значајно да постоје различите "величине" бесконачности, што представља један од најдубљих резултата теорије скупова и модерне математике.
\end{document}
