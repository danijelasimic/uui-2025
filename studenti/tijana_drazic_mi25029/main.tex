% !TEX encoding = UTF-8 Unicode
\documentclass[a4paper,12pt]{article}

\usepackage[T2A]{fontenc}
\usepackage[utf8]{inputenc}
\usepackage[serbianc]{babel}

\usepackage{amsmath,amsthm}
\usepackage{graphicx}
\usepackage{wrapfig}
\usepackage{xcolor}
\usepackage{geometry}
\usepackage{enumitem}
\usepackage{caption}
\usepackage[colorlinks=true, linkcolor=blue]{hyperref}

\geometry{margin=2.5cm}

\newtheorem{definition}{Дефиниција}
\newtheorem{theorem}{Теорема}
\newtheorem{lemma}{Лема}

\begin{document}

\begin{titlepage}
\centering

\vspace*{4cm}  

{\LARGE Математички факултет}\\[2cm]
{\Large Семинарски рад}\\[0.3cm]
{\large Увод у информатику}\\[1cm]

\rule{\linewidth}{0.4mm}\\[0.4cm]
{\LARGE\bfseries Математички модели у машинском учењу}\\[0.4cm]
\rule{\linewidth}{0.4mm}

\vspace*{\fill}  

\large
Тијана Дражић \\[0.5cm]
Београд, \today

\end{titlepage}

\tableofcontents
\newpage

\section{Увод}

Машинско учење представља област рачунарства која се у великој мери
ослања на \textbf{математичке моделе}. Један од основних и најчешће
коришћених модела је \textit{линеарна регресија}, која служи за
апроксимацију односа између података.

\section{Линеарна регресија као математички модел}

\subsection{Дефиниција модела}

\begin{definition}
Линеарна регресија је модел који описује зависност променљиве $y$
од променљиве $x$ у облику функције
\[
y = ax + b
\]
где су $a$ и $b$ реални параметри.
\end{definition}

У оквиру дефиниције модела, могу се издвојити основни кораци примене:
\begin{itemize}
    \item Формулисање математичког односа
    \item Одређивање параметара $a$ и $b$
    \item Примена модела на реалне податке
\end{itemize}

\subsection{Функција грешке}

Квалитет модела се мери функцијом грешке. За континуалне податке,
користи се \textbf{средња квадратна грешка} дефинисана интегралом:
$$ E(a,b) = \int_{-\infty}^{\infty} (ax + b - f(x))^2 \, dx $$

У пракси, подаци су дискретни, па се интеграл замењује сумом:
$$ E(a,b) = \frac{1}{n} \sum_{i=1}^{n} (ax_i + b - y_i)^2 $$


\subsection{Оптимизација параметара}

\begin{theorem}
Функција грешке линеарне регресије има јединствен глобални минимум.
\end{theorem}

\begin{lemma}
Минимум функције грешке добија се решавањем система линеарних једначина.
\end{lemma}

Ови резултати омогућавају ефикасну примену алгоритама у машинском учењу.


\section{Илустрација модела}

\noindent
\begin{minipage}[c]{0.45\textwidth}
    \centering
    \includegraphics[width=\linewidth]{slika1.png}
    \captionof{figure}{Линеарна регресија над подацима}
\end{minipage}
\hfill
\begin{minipage}[c]{0.5\textwidth}
Графички приказ регресионе праве омогућава визуелну процену
тачности модела.
\end{minipage}

\vspace{0.5cm}

\begin{center}
\begin{tabular}{|c|c|}
\hline
\textbf{Елемент} & \textbf{Улога у илустрацији} \\
\hline
$x_i$ & Тачке на хоризонталној оси (улазни подаци) \\
$y_i$ & Тачке на вертикалној оси (излазне вредности) \\
$a$ & Нагиб регресионе праве \\
$b$ & Пресек са $y$-осом \\
\hline
\end{tabular}
\captionof{table}{Елементи који се приказују на графику линеарне регресије}
\end{center}

\section{Примена модела}

Примена модела у пракси подразумева следеће кораке:
\begin{enumerate}
    \item Прикупљање података
    \item Тренирање модела
    \item Процена резултата
\end{enumerate}

\section{Закључак}

Линеарна регресија представља једноставан, али моћан математички модел
који показује како се \textbf{интеграли, суме и линеарна алгебра}
користе у савременом рачунарству и машинском учењу.

\end{document}