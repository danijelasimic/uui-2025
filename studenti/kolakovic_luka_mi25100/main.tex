\documentclass[a4paper,12pt]{article}
\usepackage[utf8]{inputenc}
\usepackage[T2A]{fontenc}
\usepackage[serbianc]{babel}
\usepackage{graphicx}
\usepackage{amsmath}
\usepackage{amssymb}
\usepackage{amsthm}
\usepackage{xcolor}
\usepackage{colortbl}
\usepackage{geometry}
\usepackage{wrapfig}

\definecolor{neonblue}{RGB}{0,245,255}
\definecolor{neonpink}{RGB}{255,0,184}
\definecolor{softgray}{RGB}{204,204,204}

\geometry{margin=2.5cm}
\arrayrulecolor{softgray}

\newtheorem{teorema}{Теорема}
\newtheorem{definicija}{Дефиниција}
\newtheorem{lema}{Лема}

\title{Анализа случаја Малоне Лам: \\ Социјални инжењеринг и прање новца}
\author{Лука Колаковић \\ Број индекса: 25100}
\date{\today}

\begin{document}

\pagecolor{black}
\color{softgray}
\ttfamily
\sloppy

\maketitle

\tableofcontents

\newpage

\section{Увод}

Случај Малоне Лам представља типичан пример \textbf{person-to-person} крађе криптовалута, у којој нападач не експлоатише техничку рањивост система већ понашање самог корисника. Циљ овог кратког рада је да, на основу јавних података, прикаже основне фазе напада, модел вероватноће успеха и основне импликације по сајбер безбедност.

Фокус је на \textit{социјалном инжењерингу} као механизму за стицање почетног приступа, затим на прању новца кроз више блокчејн мрежа и на улози формалних модела у разумевању ризика.

\section{Теоријски оквир}

\subsection{Социјални инжењеринг}

\begin{definicija}
Социјални инжењеринг је контролисана психолошка манипулација људи са циљем да открију поверљиве информације или изврше радње које иду у корист нападача, а на штету сопствене безбедности.
\end{definicija}

У контексту криптовалута, циљ нападача је да дође до \textit{приватних кључева} или других акредитива (налог на берзи, резервне копије новчаника). Техничке мере заштите су често снажне, али корисник остаје најслабија карика.

\subsection{Модел вероватноће успеха}

Нека је $P(s)$ вероватноћа успеха напада, $n$ број покушаја контакта са метом (телефонски позиви, поруке), а $I$ мерa квалитета информација о мети (контакт подаци, навике, финансијски профил). Једноставан модел гласи

\begin{equation}
    P(s) = 1 - e^{-k \cdot n \cdot I},
\end{equation}

где је $k \!>\! 0$ параметар ефикасности нападачког тима.

\begin{teorema}
За фиксно $k$, вероватноћа успеха $P(s)$ је монотоно растућа у односу на $n$ и $I$.
\end{teorema}

\begin{lema}
Ако је $I = 0$, односно нападач нема никакве информације о мети, тада је $P(s) = 0$ без обзира на број покушаја $n$.
\end{lema}

Ови резултати формално потврђују интуицију да су систематично прикупљање информација и упорност кључни фактори успеха напада.

\section{Анализа случаја Малоне Лам}

\subsection{Фазе напада и прања новца}

У јавним материјалима о овом случају може се уочити неколико типичних фаза:

\begin{enumerate}
    \item иницијални контакт и представљање као техничка подршка (нпр. \textit{Google} или крипто берза),
    \item стицање контроле над налозима корисника (мејл, облак, берза),
    \item пренос средстава на новчанике које контролише нападач,
    \item прање новца кроз више платформи и криптовалута.
\end{enumerate}

Прање новца често укључује тзв. \textit{chain hopping}, односно секвенцијалну конверзију из једне криптовалуте у другу:

\begin{itemize}
    \item \textcolor{neonpink}{Bitcoin} $\rightarrow$ \textcolor{neonblue}{Ethereum},
    \item \textcolor{neonblue}{Ethereum} $\rightarrow$ Litecoin,
    \item Litecoin $\rightarrow$ \textcolor{neonblue}{Monero}, криптовалута са појачаним механизмима приватности.
\end{itemize}

У поједностављеној табеларној форми, ток средстава се може приказати као:

\begingroup
\color{white}
\begin{table}[h!]
\centering
\begin{tabular}{|>{\color{white}}l|>{\color{white}}l|>{\color{white}}r|}
\hline
\textbf{Фаза} & \textbf{Опис} & \textbf{Приближни износ (USD)} \\ \hline
1 & Иницијална крађа са берзе & 4\,500\,000 \\ \hline
2 & Касније трансакције и допуне & 200\,000\,000+ \\ \hline
3 & Укупно опрана средства & 230\,000\,000 \\ \hline
4 & Заплењена имовина (FBI) & 59\,000\,000 \\ \hline
\end{tabular}
\end{table}
\endgroup

\subsection{Безбедносне импликације}

Случај Малоне Лам показује да и релативно зрела инфраструктура као што је \textbf{blockchain} није довољна ако су процеси аутентикације и верификације корисника слабо дефинисани. Посебно је критично ослањање на један канал комуникације (телефон) и на неформалне процедуре ресетовања налога.

\section{Закључак}

У овом кратком раду приказан је сажет теоријски оквир за разумевање \textit{person-to-person} напада на криптовалуте, као и примена тог оквира на случај Малоне Лам. Комбинација социјалног инжењеринга и вишестепеног прања новца доводи до напада велике финансијске вредности, при чему људски фактор остаје централна слабост.

\begin{wrapfigure}{l}{0.4\textwidth}
    \centering
    \includegraphics[width=0.35\textwidth]{slike/crypto_exchange_logo.png}
    \caption{Лого крипто менјачнице Gemini}
\end{wrapfigure}

Практична поука је да мере заштите морају обухватити не само техничке механизме (попут шифровања), већ и \textbf{стандардизоване процедуре} за рад са корисницима, обуку запослених и континуирано праћење аномалија у понашању налога.

\end{document}
