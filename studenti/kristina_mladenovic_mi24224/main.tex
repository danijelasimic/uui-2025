\documentclass[a4paper,12pt]{article}
\usepackage[utf8]{inputenc} 
\usepackage[T2A]{fontenc}    
\usepackage[serbian]{babel}
\selectlanguage{serbian}
\addto\captionsserbian{\renewcommand{\contentsname}{Садржај}}
\renewcommand{\contentsname}{Садржај}
\usepackage{graphicx}       
\usepackage{amsmath, amsthm} 
\usepackage{xcolor}         
\usepackage{geometry}        
\usepackage{float}
\usepackage{amsthm}

\newtheorem*{theorem}{Теорема}
\newtheorem*{lemma}{Лема}
\newtheorem*{definition}{Дефиниција}


\geometry{left=2.5cm, right=2.5cm, top=2.5cm, bottom=2.5cm}

\definecolor{softpink}{RGB}{214, 51, 132}
\usepackage[colorlinks=true, linkcolor=softpink]{hyperref}

\title{\textbf{Соларна енергија и њена примена}}
\author{Кристина Младеновић}
\date{Јануар 2026.}



\begin{document}

\maketitle

\tableofcontents
\newpage

\section{Увод}
Соларна енергија представља \textbf{обновљив извор енергије} који се добија претварањем Сунчевог зрачења у електричну или топлотну енергију. 
Растућа потреба за одрживим изворима енергије чини соларне системе све значајнијим у савременом друштву. 
Овај рад има за циљ да прикаже основне карактеристике соларне енергије и њене примене.
\emph{Соларна енергија} се сматра једним од најперспективнијих извора енергије у будућности.

\textcolor{softpink}{Одрживи развој} представља кључни циљ савремених енергетских политика.



\section{Основни појмови}
\subsection{Дефиниција}
\begin{definition}
Соларни панел је уређај који претвара Сунчеву светлост у електричну енергију коришћењем фотонапонских ћелија.
\end{definition}

\subsection{Теорема и лема}
У наставку су наведене основне тврдње које описују понашање соларних панела.
\begin{theorem}
Ефикасност соларних панела зависи од угла сунчевог зрачења, интензитета светлости и температуре.
\end{theorem}

\begin{lemma}
Повећање површине панела доводи до пропорционалног повећања произведене електричне енергије.
\end{lemma}

\subsection{Формула}
Једна од најпознатијих физичких формула која описује однос масе и енергије је:
\[
E = mc^2
\]
где је \(E\) енергија, \(m\) маса, а \(c\) брзина светлости.

\section{Предности и примена соларне енергије}
\subsection{Предности}
Неколико кључних предности соларне енергије укључује:
\begin{itemize}
    \item Обновљивост и доступност
    \item Смањење емисије CO$_2$
    \item Дугорочна економска исплативост
\end{itemize}

\subsection{Области примене}
\begin{enumerate}
    \item Домаћинства – производња електричне енергије
    \item Индустрија – смањење трошкова енергије
    \item Јавне установе – осветљење и грејање
\end{enumerate}

\subsection{Табела примера примене}

\begin{table}[H]
\centering
\begin{tabular}{|l|l|}
\hline
Област & Намена \\ \hline
Домаћинства & Производња електричне енергије \\ \hline
Индустрија & Смањење трошкова енергије \\ \hline
Јавне установе & Осветљење и грејање \\ \hline
\end{tabular}
\caption{Примена соларне енергије у различитим областима}
\end{table}

\subsection{Илустрација соларног система}

\begin{figure}[H]
\centering
\includegraphics[width=0.35\textwidth]{slike/solar.png}
\hfill
\parbox[b]{0.6\textwidth}{
На слици је приказан пример фотонапонског система са соларним панелима.
Соларни панели апсорбују сунчево зрачење и претварају га у електричну енергију
помоћу фотонапонских ћелија. Овакви системи се широко примењују у домаћинствима,
индустрији и јавним установама ради смањења потрошње енергије из
необновљивих извора.}
\caption{Пример фотонапонског соларног система}
\end{figure}


\section{Закључак}
Соларна енергија је одржив и чист извор енергије који има значајан потенцијал у свакодневној употреби. 
Коришћењем соларних панела могуће је смањити емисију штетних гасова и дугорочно смањити трошкове електричне енергије.
Овај рад демонстрира примену основних LaTeX елемената: формула, табела, листа, слика, теорема и дефиниција.

\end{document}
