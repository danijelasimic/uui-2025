\documentclass[12pt,a4paper]{article}
\usepackage[T2A]{fontenc}
\usepackage[utf8]{inputenc}
\usepackage[serbian]{babel}
\usepackage{graphicx}

\graphicspath{{slike/}}

% како ћирилично написати овај наслов...?
\title{\textbf{XYZ колор систем}}
\author{Јована Перуничић}
\date{15.02.2026.}

\begin{document}
\maketitle
\newpage
\tableofcontents
\newpage

\section{Колор системи}

Колор систем је специфична организација боја дефинисана приманим компонентама помоћу којих можемо рекреирати боје. Може бити аналогна или дигитална репрезентација, зависна или независна од уређаја, имати особине униформности итд.
Неки битнији колор системи:
\begin{enumerate}
\item Зависни од уређаја:
\begin{itemize}
\item RGB
%\item YCbCr
\item HSV, HSL, HSI
\end{itemize}
\item Независни од уређаја:
\begin{itemize}
\item XYZ
\item Lab
%\item LCh
\end{itemize}
\end{enumerate}

\section{Настанак и потреба за XYZ колор системом}

Током 1920-их рађена су испитивања са великим бројем посматрача ради формирања конкретне дефиниције RGB колор система. Она су се ослањала на анализу мешања и поређења боја посматраних голим оком. Математичким упросечавањем ових резултата добиле су се \textit{криве за мешање боја стандардног посматрача}.

\begin{figure}[h!]
\centering
\includegraphics[width=0.70\textwidth]{krive1.png}
\caption{CIE RGB криве за мешање боја стандардног посматрача}
\end{figure}
\pagebreak

\subsection{Проблематика RGB колор система}

Водећа претпоставка у оваквим системима је да референтна светлост има равномерну расподелу енергије. Количине примара потребне за постизање одређене боје се не мере директно, већ у односу на фиктивног стандардног посматрача.
На претходној слици може се уочити негативни део црвеног примара. Он не представља \textit{негативну светлост}, већ последице математичке еквиваленције примарних извора светлости (R,G, и B). Ово указује на могућност појављивања изобличења у репродукцији слике.

\subsection{Увођење XYZ колор система}

У циљу избегавања компликација са негативним вредностима приликом мешања боја, уводи се нови систем, представљен са X, Y, и Z примарима, чије криве имају само позитивне вредности.

\begin{figure}[h!]
\centering
\includegraphics[width=0.70\textwidth]{krive2.png}
\caption{CIE XYZ криве за мешање боја стандардног посматрача}
\end{figure}

Ови примари не представљају стварне боје и не могу се физички приказати, већ чине математичке апроксимације којима се могу описати све боје светлости.

\section{Преглед XYZ колор система}

Поред горе наведеног, битно је навести да у овом систему само два примара одређују боју (X, Z), док трећи (Y) одређује осветљеност.
На овај начин се са просторног приказа дијаграма боја прешло на површински, који називамо \textbf{потковичаста спектрална крива боја}.

\begin{figure}[h!]
\centering
\includegraphics[width=0.5\textwidth]{potkovica.png}
\caption{Стандардни дијаграм боја CIE система}
\end{figure}

Међународна комисија за осветљење (CIE) је дефинисала и нормализоване вредности примара, који се обележавају са x, y, z, где је сјајност референтне беле светлости нормализована на 1:

\[ x + y + z = 1 \]

Нормализоване вредности примара се добијају следећим формулама:

\[ x = \frac{X}{(X + Y + Z)} \]
\[ y = \frac{Y}{(X + Y + Z)} \]
\[ z = \frac{Z}{(X + Y + Z)} \]

\section{Закључак}

CIE XYZ колор систем је представља основни координатни систем за представу боја незавсно од уређаја и данас и из њега су изведени бројни други колор системи.
За крај, могу се видети и различити стандарди телевизијске слике.

\begin{table}[h]
\centering
\begin{tabular}{||c|c c c||}
\hline
\textbf{NTSC} & \textbf{RED} & \textbf{GREEN} & \textbf{BLUE} \\
\hline
(1953) & Xr  Yr & Xg  Yg & Xb  Yb \\
\hline
   & 0.670  0.330 & 0.210  0.710 & 0.140  0.080 \\
\hline
Illuminat C & x = 0.3101 & y = 0.3162 &   \\
\hline
\textbf{EBU} & \textbf{RED} & \textbf{GREEN} & \textbf{BLUE} \\
\hline
Rec 709 & Xr  Yr & Xg  Yg & Xb  Yb \\
\hline
   & 0.640  0.330 & 0.300  0.600 & 0.180  0.060 \\
\hline
Illuminat D65 & x = 0.3127 & y = 0.3290 &   \\
\hline
\textbf{PAL/SECAM} & \textbf{RED} & \textbf{GREEN} & \textbf{BLUE} \\
\hline
    & Xr  Yr & Xg  Yg & Xb  Yb \\
\hline
   & 0.640  0.330 & 0.290  0.600 & 0.150  0.060 \\
\hline
Illuminat D65 & x = 0.3127 & y = 0.3290 &   \\
\hline
\textbf{SMPTE} & \textbf{RED} & \textbf{GREEN} & \textbf{BLUE} \\
\hline
    & Xr  Yr & Xg  Yg & Xb  Yb \\
\hline
   & 0.630  0.340 & 0.310  0.595 & 0.155  0.070 \\
\hline
Illuminat D65 & x = 0.3127 & y = 0.3290 &   \\
\hline
\end{tabular}
\caption{Приказ CIE координата (x,y) за различите ТВ стандарде}
\end{table}
\end{document}
