\documentclass[a4paper]{article}
\usepackage{graphicx} 
\usepackage[english,serbianc]{babel}
\usepackage[T2A]{fontenc}
\usepackage[utf8]{inputenc}
\usepackage{color}
\usepackage{url}
\usepackage{amsthm}

\newtheorem{theorem}{Теорема}
\newtheorem{definition}{Дефиниција}

\begin{document}

\begin{titlepage}

\newcommand{\HRule}{\rule{\linewidth}{0.4mm}}

\center 
	
\textsc{\LARGE Математички факултет}\\[3cm] 
	
\textsc{\Large Семинарски рад}\\[0.1cm]
	
\textsc{\large из увода у информатику}\\[0.5cm] 
	
\HRule\\[0.4cm]
	
{\LARGE\bfseries Хешов број}\\[0.2cm] 
	
\HRule\\[2cm]

\vspace{17\baselineskip}
		

	
	\begin{minipage}{0.4\textwidth}
		\begin{flushleft}
			\large
			\textit{Студент}\\
			Александра Попов 86/2025
		\end{flushleft}
	\end{minipage}
\hspace*{1cm}
	\begin{minipage}{0.4\textwidth}
		\begin{flushright}
			\large
			\textit{Професор}\\
			Данијела Симић
		\end{flushright}
	\end{minipage}
	
	
\vfill\vfill\vfill\vfill 
	
{\large Београд, фебруар 2026.} 
	
\vfill 
	
\end{titlepage}

\tableofcontents

\newpage
\abstract{
Хешов проблем се бави проналажењем фигура или тела која могу да поплочају раван искључиво за тачно одређен број слојева.\\ \\
Са овом темом сам се први пут сусрела у Истраживачкој станици Петница, где сам, уз менторство Бојана Башића, проучавала и презентовала његов рад посвећен овом проблему. Посебно је значајно истаћи да је он пронашао фигуру са Хешовим бројем 6.
}
\section{Увод}
\textbf{Хешов број} представља број слојева колико пута се неки облик $(D^2)$ или тело $(D^n)$ могу описати око себе самих, тако да не долази ни до каквих преклапања или празних простора, то јест „рупа“ између. Полазно тело или облик имају Хешов број 0, а сваки наредни слој се рачуна као број слојева који је могуће описати око почетне фигуре.  
\section{Теселације}
\textbf{Теселација}, односно поплочавање, је појам који означава покривање површине геометријским облицима без преклапања и без празнина. \\ 
Постоје теселације: 
\begin{itemize}
\item Еуклидске $(E^2)$,
\item Сферне $(S^2)$
\item Хиперболичке $(H^2)$ равни.
\end{itemize}

Дакле, постоји  могућност теселације у више димензија.\\
Поплочавање се може вршити помоћу три операције:
\begin{enumerate}
\item Ротације,
\item Транслације,
\item Рефлексије.
\end{enumerate}

\section{Хешов број}
\begin{definition}
Ако скуп плочица може да поплоча раван, његов Хешов број је бесконачан. Ако не може, број је коначан и представља максималан број слојева плочица које могу да окружују почетну плочицу пре него што се појаве празнине. 
\end{definition}
 На пример, квадрат можете да слажете и правите нове слојеве поплочавајући раван у бесконачност, што значи да квадрат има Хешов број бесконачно много. За разлику од квадрата, око круга никако не можете направити ни један слој без преклапања или празнина између, тј. не можете поплочати раван. Самим тим Хешов број круга је 0 - нула представља само почетни облик, без других слојева. На исти начин се добијају и Хешови бројеви фигура у $(D^3)$. Поплочавање се усложњава када дођемо до већих димензија ( од 4 па навише). У том случају се служимо хипертелима, најчешће хиперкоцкама.

\begin{table}[h!]
\begin{center}
\caption{Историјски преглед открића Хешових бројева.}
\label{tab:табела}
\begin{tabular}{|c|c|c|} \hline
Хешов број& Година проналаска фигуре& Проналазач\\ \hline
1&1928 / 1968&Walther Lietzmann / Heinrich Heesch\\ \hline
2&1991&Anne Fontaine\\ \hline
3&1990-1995&Robert Ammann\\ \hline
4&2000-те &Alex Day / R. A. Marshall\\ \hline
5&2001&Casey Mann\\ \hline
6&2020/2021&Бојан Башић\\ \hline
\end{tabular}
\end{center}
\end{table}

\begin{figure}[h!]
\begin{center}
\includegraphics[scale=0.7]{hesbrsest.jpg}
\caption{Фигура коју је открио Бојан Башић, и њених шест слојева (Сваки слој је приказан једном бојом; почетна фигура је циве боје).}
\label{fig:хешбршест}
\end{center}
\end{figure}


\subsection{Хешов проблем – отворено питање}
\begin{theorem}
Да ли за сваки позитиван цео број n постоји плочица T чији је Хешов број тачно n?
\end{theorem}
До краја 20. века, било је познато само неколико примера плочица са коначним Хешовим бројем, а тек новија истраживања успела су да конструишу плочице са произвољно великим коначним Хешовим бројевима, чиме је омогућен даљи напредак у разумевању ограничења и потенцијала различитих облика за поплочавање равни.

\section{Закључак}
Хешов проблем и даље представља изазов за математичаре, упркос бројним доказима. Са развојем нових приступа постаје све актуелнији, а неки га истражују и из радозналости, креирајући занимљиве моделе плочица. Моје интересовање за ову тему се јавило већ при првом сусрету, иако тада нисам имала предзнања о теселацијама. Проблем је довољно интуитиван иако нисте у научним водама, а истовремено довољно сложен за озбиљна научна истраживања.

\end{document}
