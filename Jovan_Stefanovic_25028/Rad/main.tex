\documentclass[12pt,a4paper]{article}

% --------------------
% PODRŠKA ZA ĆIRILICU (pdflatex)
% --------------------
\usepackage[T2A]{fontenc}
\usepackage[utf8]{inputenc}
\usepackage[serbianc]{babel}

% --------------------
% PAKETI
% --------------------
\usepackage{amsmath, amsthm}
\usepackage{graphicx}
\usepackage{float}
\usepackage{enumitem}
\usepackage{wrapfig}

% --------------------
% TEOREME I DEFINICIJE
% --------------------
\newtheorem{teorema}{Теорема}
\newtheorem{lema}{Лема}
\theoremstyle{definition}
\newtheorem{definicija}{Дефиниција}

% --------------------
% PODACI O RADU
% --------------------
\title{\textbf{Поређење временске сложености алгоритама сортирања}}
\author{Јован Стефановић}
\date{\today}

\begin{document}

\maketitle
\tableofcontents
\newpage

% --------------------
\section{Увод}

Иако више алгоритама може успешно сортирати податке,
њихова ефикасност може значајно да се разликује.
Због тога је временска сложеност један од
\textbf{најважнијих критеријума} при избору алгоритма сортирања.

У овом раду анализира се начин рада најчешћих алгоритама сортирања,
као и њихова асимптотска временска сложеност.

% --------------------
\section{Основни појмови}

\begin{definicija}
Временска сложеност алгоритма је функција $T(n)$
која описује како број основних операција зависи
од величине улаза $n$.
\end{definicija}

Код алгоритама сортирања,
основна операција је најчешће \textbf{поређење елемената}.
За упоређивање алгоритама користи се Big-O нотација:
\[
T(n) = O(f(n))
\]

% --------------------
\section{Алгоритми квадратне сложености}

\subsection{Сортирање балончићима}

Сортирање балончићима је једноставан алгоритам који више пута пролази кроз низ
и упоређује суседне елементе, мењајући им места ако нису у исправном редоследу.
Овај поступак се понавља све док низ не буде потпуно сортиран.

Број поређења у најгорем случају износи:
\[
C(n) = \frac{n(n-1)}{2}
\]

што доводи до временске сложености:
\[
T(n) = O(n^2)
\]

\subsection{Сортирање одабиром}

Сортирање одабиром у сваком кораку проналази најмањи елемент
у несортираном делу низа и поставља га на одговарајућу позицију.
Број поређења не зависи од почетног распоредa елемената.

Због тога алгоритам увек има сложеност:
\[
T(n) = O(n^2)
\]

\subsection{Сортирање уметањем}

Сортирање уметањем гради сортирани део низа тако што
сваки нови елемент убацује на одговарајуће место,
слично начину на који се карте ређају у руци.

У најбољем случају (већ сортиран низ) важи:
\[
T(n) = O(n)
\]

док је у најгорем случају:
\[
T(n) = O(n^2)
\]

% --------------------
\section{Алгоритми логаритамске сложености}

\subsection{Сортирање спајањем}

Сортирање спајањем је алгоритам типа \textit{подели па владај}.
Низ се рекурзивно дели на две половине
док се не добију низови дужине један,
који се затим спајају у сортиран редослед.

\begin{lema}
Временска сложеност сортирања спајањем алгоритма
може се описати релацијом:
\[
T(n) = 2T\left(\frac{n}{2}\right) + n
\]
\end{lema}

Решавањем ове релације добија се:
\[
T(n) = O(n \log n)
\]

\subsection{Брзо сортирање}

Брзо сортирање такође користи приступ \textit{подели па владај}.
Алгоритам бира један елемент (пивот)
и распоређује остале елементе тако да су мањи лево,
а већи десно од пивота.

У просечном случају важи:
\[
T(n) = O(n \log n)
\]

Међутим, у најгорем случају (лош избор пивота):
\[
T(n) = O(n^2)
\]

% --------------------
\section{Теоријска доња граница}

\begin{teorema}
Ниједан алгоритам сортирања заснован искључиво на поређењу
не може имати временску сложеност бољу од
$O(n \log n)$ у најгорем случају.
\end{teorema}

Ова теорема показује да алгоритми попут Merge sort-а
имају асимптотски оптималну сложеност.

% --------------------
\section{Табеларно поређење}

\begin{table}[H]
\centering
\begin{tabular}{|c|c|c|c|}
\hline
\textbf{Алгоритам} & \textbf{Најбољи} & \textbf{Просечан} & \textbf{Најгори} \\
\hline
Сортирање балончићима & $O(n)$ & $O(n^2)$ & $O(n^2)$ \\
Сортирање Уметањем & $O(n)$ & $O(n^2)$ & $O(n^2)$ \\
Брзо сортирање & $O(n \log n)$ & $O(n \log n)$ & $O(n^2)$ \\
Сортирање спајањем & $O(n \log n)$ & $O(n \log n)$ & $O(n \log n)$ \\
\hline
\end{tabular}
\caption{Поређење временске сложености алгоритама}
\end{table}

% --------------------
\section{Илустрација раста сложености}

\begin{wrapfigure}{l}{0.4\textwidth}
    \centering
    \includegraphics[width=0.38\textwidth]{sorting.png}
    \caption{Поређење раста функција}
\end{wrapfigure}

Функције $n^2$ и $n \log n$ имају различито понашање
за велике вредности $n$,
што објашњава значај избора ефикасног алгоритма.

\par
% --------------------

\section{Закључак}

Овај рад показује значајну разлику између различитих алгоритама
за сортирање и придодаје значај одабиру истих.

Иако једноставни алгоритми имају образовну вредност,
алгоритми сложености $O(n \log n)$
представљају практичан избор у реалним применама.

\end{document}
